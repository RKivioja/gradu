\documentclass[utf8]{gradu3}

\usepackage{graphicx}
\usepackage{amsmath}
\usepackage{booktabs}
\usepackage[backend=biber]{biblatex}

% HUOM! Tämän tulee olla viimeinen \usepackage koko dokumentissa!
\usepackage[bookmarksopen,bookmarksnumbered,linktocpage]{hyperref}

\addbibresource{gradubibtex.bib}

\begin{document}

\title{Puheen ongelmista kärsiville tarkoitetun kommunikointisovelluksen toteuttaminen Ionic-kehyksellä}
\translatedtitle{Puheen ongelmista kärsiville tarkoitetun kommunikointisovelluksen toteuttaminen Ionic-kehyksellä}
\studyline{Ohjelmistotekniikka}
\avainsanat{%
  Ionic,
  AAC,
  www-sovellukset,
  käytettävyys
}
\keywords{%
  Ionic,
  AAC,
  web applications,
  usability
}
\tiivistelma{%
  Tutkielman tiivistelmä on tyypillisesti lyhyt esitys, jossa kerrotaan tutkielman taustoista, tavoitteesta, tutkimusmenetelmistä, saavutetuista tuloksista, tulosten tulkinnasta ja johtopäätöksistä. Tiivistelmän tulee olla niin lyhyt, että se, englanninkielinen abstrakti ja muut metatiedot mahtuvat kaikki samalle sivulle.
}
\abstract{%
  Tiivistelmä englanniksi.
}

\author{Roope Kivioja}
\contactinformation{\texttt{roope.kivioja@gmail.com}}

\supervisor{Jukka-Pekka Santanen}
\supervisor{Jonne Itkonen}

\maketitle

\begin{thetermlist}
\item[Ionic] Ohjelmistokehys.
\end{thetermlist}

\mainmatter

\chapter{Johdanto}

Tähän tulee johdanto.

\chapter{Kommunikaatiossa avustavan sovelluksen rakentamiseen tarvittavia taustatietoja ja menetelmiä}

\section{Avusteinen kommunikaatio}

- TODO: Kerrotaan yleisesti taustoista (mihin avustavaa tietotekniikkaa tarvitaan, miten sitä käytetään, mitä ongelmia jne.)

On olemassa useita eri sairauksia ja kehityshäiriöitä, joiden johdosta henkilön puheen tuottamisen kyky voi hetkellisesti tai pysyvästi heikentyä. Puheen tuottamisen ongelmista kärsivä henkilö saattaa joutua turvautumaan arkipäivän kommunikaatiossa erilaisiin apuvälineisiin kommunikoidakseen muiden ihmisten kanssa. Yleisesti näitä viestintämenetelmiä kuvaamaan tarkoitettu termi on "puhetta tukeva ja korvaava kommunikaatio" (engl. Augmentative and Alternative Communication, lyh. AAC).

Puhetta tukeva ja korvaava kommunikaatio voidaan jakaa kahtia: avustamattomaan ja avusteiseen. Avustamattomalla puhetta tukevalla ja korvaavalla kommunikaatiolla tarkoitetaan puhetta tukevaa ja korvaavaa kommunikaatiota, jossa ei tarvita apuvälineitä. Viittomankieli on yksi esimerkki avustamattomasta puhetta tukevasta ja korvaavaasta kommunikaatiosta, mutta myös ihmisen elekieltä voidaan pitää avustamattomana puhetta tukevana ja korvaavana kommunikaationa. 

Avustettu puhetta tukevaa ja korvaava kommunikaatio tarkoittaa puolestaan kommunikaatiota, jossa käytetään jotain apuvälinettä. Apuvälineenä voi olla esimerkiksi valokuvia, kommunikaatiotaulu tai elektroninen laite. Tämän perusteella avustettu puhetta tukevaa ja korvaava kommunikaatio voidaan jakaa vielä eteenpäin kahdeksi ryhmäksi: matalan teknologian ja korkean teknologian puhetta tukevaan ja korvaavaan kommunikaatioon. Käyttäjä ei välttämättä käytä vain yhtä edellä mainituista tyypeistä. \parencite[]{AAC-conditional-use}

Puhetta tukevaa ja korvaavaa kommunikaatiota tarjotaan puheen tuottamisen ongelmista kärsiville myös tietoteknisten sovellusten avulla, joiden kautta käyttäjä kirjoittaa joko suoraan tekstiä tai kommunikoi valitsemalla symboleja. Esimerkiksi autistisilla käyttäjillä on tyypillisesti hyvin henkilö- ja yksityiskohtaisia tarpeita puhetta tukevalle ja korvaavalle kommunikaatiolle, joten tietoteknisten sovellusten suhteellisen helppo muokattavuus puoltaa niiden käyttöä perinteisempien puhetta ja kommunikointia korvaavien apuvälineiden sijaan.

Puhetta tukevat ja korvaavat kommunikointisovellukset käyttävät yleisimmin symboleja. Autisteille tyypillistä on vahva visuaalis-avaruudellinen hahmotuskyky, joten tutkimuksen mukaan piirroksiin ja valokuviin liitetyt merkitykset ovat tälle ryhmälle luontevin tapa kommunikoida. Vaughnin ja Hornerin tutkimuksessa  \parencite[]{concrete-versus-verbal} Karl-nimisen autistisen koehenkilön haastava käytös ja aggressio vähenväti kun pelkästään verbaalisesti annettujen ruokavaihtoehtojen rinnalle tuotiin kuvat ruoka-annoksista. Symbolipohjaiselle puhetta tukevalle ja korvaavalle kommunikaatiolle on siis sekä tutkimuspohjaista näyttöä, että käytännön kokemuksiin perustuvaa kannustetta.

Puhetta tukevat ja korvaavat kommunikointisovellukseen voidaan liittää myös symboleita lukeva ääniominaisuus. Ääniominaisuus mahdollistaa kommunikoinnin näköyhteyden ulkopuolelle, vähentää symbolien tulkitsijan läsnäolon pakollisuutta ja helpottaa pidempien viestien rakentamista puhetta tukevassa ja korvaavassa kommunikointisovelluksessa.\parencite[]{AAC-interventions}

Nykyisten puhetta tukevien ja korvaavien kommunikointisovellukseen ongelma on, että niitä voidaan käyttää vain yhdellä laitteella (lähde?), joten progressiivisten sovellusten käyttäminen voisi parantaa tilannetta.

\section{Progressiiviset WWW-sovellukset}

- TODO: Kerrotaan yleisesti PWA-sovelluksista (mihin pohjautuu, miksi käytetään, miksi juuri tähän projektiin)

Progressiiviset sovellukset ovat selaimessa ajettavia WWW-sovelluksia, joiden ulkoasu on optimoitu laitteelle niin, että ne eivät eroa natiiveista sovelluksista. Progressiivisella sovelluksella tarkoitetaan sovellusta, joka pystyy käyttämään tarjolla olevan sovellusympäristön resursseja joustavasti sen sijaan, että se itse määrittäisi vaatimukset. Termi on verrattain tuore ja vakiintumaton, sillä esimerkiksi Ionic-kehyksen dokumentaatiossa on käytetty aiemmin myös termiä "hybridisovellus".

Tällä hetkellä erityisesti Google panostaa progressiivisten sovellusten kehittämiseen Chrome-selaimen kehitysympäristön yhteydessä. Googlen mukaan progressiiviset sovellukset tarjoavat perinteisiin WWW-sovelluksiin verrattuna enemmän luotettavuutta, käytettävyyttä ja monipuolisempaa sisältöä. Luotettavuus paranee, sillä progressiiviset sovellukset pystyvät tarjoamaan sisältöä myös ilman verkkoyhteyttä. Käytettävyyttä parantaa nopeammin käyttäjän komentoihin vastaava käyttöliittymä. Sovellusmaisuus puolestaan parantaa käyttäjän immersiota. \parencite[]{google-pwa-marketing}
TODO: tähän omaa pohdintaa

Idea natiivien sovellusten rakentamisesta selainalustoja hyödyntämällä ei ole uusi. Muutamia esimerkkejä idean käytöstä ovat Adobe AIR - ja Electron -sovellukset. Twitter, AliExpress ja Lancôme ottivat vuonna 2017 käyttöön progressiiviset sovellukset ja julkaistut tulokset ovat rohkaisevia. Twitter onnistui uuden progressiivisen sovelluksensa ansiosta lisäämään sivupäivityksiä 65\% per sessio, lisäämään lähetettyjen Twitter-viestien määrää 75\% ja vähentämään käytön lopettamista 20\%. Lisäksi uusi sovellus käytti vähemmän kuin 3\% natiivin sovelluksen vaatimasta tilasta ja vähensi datan käyttöä 70\%. Datan käytön määrän vähentyminen on erityisen merkittävää, koska Twitter arvioi, että 45\% sen sisällöstä ladataan 2G-verkon läpi. \parencite[]{beginners-guide-pwa}

AliExpress ja Lancôme ovat molemmat verkkokauppoja, joilla on ollut ongelmia mobiiliverkkokauppojensa tehokkuuden kanssa. Progressiiviseen sovellukseen siirtymällä AliExpress lisäsi uusien asiakkaiden myyntitapahtumien määrää 104\%:lla ja Lancôme 17\%:lla. Lisäksi AliExpress tuplasi sivunlatausten määrän sekä kasvatti sessioiden pituutta 74\%:lla. Lancôme puolestaan onnistui lisäämään iOS-sessioiden määrää 53\%:lla. \parencite[]{beginners-guide-pwa}  TODO: omaa pohdintaa

Progressiivisten sovellusten käyttö ei kuitenkaan ole ongelmatonta. Ensinnäkin, koska kyse on uudesta suuntauksesta ohjelmistokehitysessä, vaaditaan kehittäjiltä paljon uusien asioiden opettelua sekä vakiintumattomien työkalujen sekä kehysten käyttämistä kehitystyössä. Toinen merkittävä kompastuskivi voi olla tallentaminen, sillä selaimen toimintaan pohjautuvalle sovellukselle ei ole varattu luontevaa tallennustilaa. Kolmas ongelma voi tulla eteen käytettävyydessä, sillä selaimessa toimivan sovelluksen vaatimat resurssit saattavat hidastaa sovelluksen toimintaa. TODO: tähän omaa pohdintaa \parencite[]{pwa-design-challenges}

\section{Ionic}

Ionic on yksi suosituimpia progressiivisten sovellusten tuottamiseen suunnitelluista kehyksistä. Ionic on avointa lähdekoodia ja hyödyntää Apache Cordova -ympäristöä sekä Angular-ohjelmistokehystä. Ionic on MIT-lisenssin alainen. Angular-ohjelmistokehys perustuu TypeScript-ohjelmointikieleen. Kehyksenä Ionicin pääpaino on tarjota kehittäjälle oikealta näyttävä ja visuaalisesti toimiva sovellus. Sen ei ole tarkoitus korvata tyypillisiä JavaScript-kirjastoja vaan se toimii niiden tukena. \parencite[]{ionic-documentation}

Ionic koostuu komponenteista. Ionicin komponentit ovat uudelleenkäytettäviä käyttöliittymäelementtejä, jotka toimivat sovelluksen käyttöliittymän rakennusosina. Niiden avulla sovellus näyttää yhtenäiseltä ja käyttöliittymän kehitystyö nopeutuu. Komponentit koostuvat HTML:stä, CSS:stä ja JavaScriptistä.

Ionicin ulkoasu perustuu teemoihin. Myös teemat lisäävät sovelluksen ulkoasun yhtenäisyyttä. Teemat myös mukautuvat eri alustojen ulkoasustandardien mukaisiksi kehittäjän niin halutessa. Esimerkiksi Android- ja iOS -mobiilikäyttöjärjestelmille suunnattujen sovellusten ulkoasu poikkeaa toisistaan, jos käytetään Ionicin oletusteemoja. Teemojen käyttäminen parantaa sovelluksen käytettävyyttä tekemällä sen ulkoasusta ennustettavamman ja tutumman loppukäyttäjälle.

 -> katso dokumentaatiosta

\subsection{Angular}

Angular on Googlen Angular Teamin ylläpitämä TypeScript-pohjainen käyttöliittymäkehys. Se on jatkoa aiemmin ilmestyneelle AngularJS-kehykselle. Alkuperäinen AngularJS-kehys ilmestyi vuonna 2010 ja se oli ensimmäinen valmis ratkaisu dynaamisten HTML-sivujen rakentamiseen mahdollistaen tehokkaamman yhden sivun WWW-sovellusten rakentamisen. Angular julkaistiin vuonna 2016 ja se on kokonaan uudelleenohjelmoitu versio AngularJS:stä.

AngularJS perustuu MVC-arkkitehtuuriin kun taas uudempi Angular on komponenttipohjainen. Komponenttipohjainen arkkitehtuuri pyrkii tarjoamaan paremman uudelleenkäytettävyyden, luettavuuden, testattavuuden ja ylläpidettävyyden. Ohjelma jaetaan itsenäisiin komponentteihin, jota voidaan käyttää useasti ja niiden itsenäisyys helpottaa yksikkötestaamista. Itsenäiset komponentit ovat huomattavasti helpommin ymmärrettävissä ja niiden korvaaminen on joustavampaa mikä parantaa ylläpidettävyyttä. \parencite[]{good-and-bad-angular}

\subsection{TypeScript}

Angular-ohjelmointikehyksen avulla ohjelmointiin käytetään TypeScript-ohjelmointikieltä. TypeScript on avoimen lähdekoodin ohjelmointikieli, jota ylläpitää Microsoft. Se kääntyy suoritusvaiheessa JavaScriptiksi ja se on tarkoitettu tukemaan JavaScript-ohjelmien kehitystä tarjoamalla muunmuassa staattisen tyypityksen. 

TypeScriptin on tarkoitus tarjota Microsoftin .NET-ympäristöön tottuneille ohjelmoijille oliolähtöisempi lähestymistapa JavaScript-kehitykseen \parencite[]{maharry-typescript}. JavaScript on suosittu ohjelmointikieli, mutta varsinkin ohjelmiston lähdekoodin määrän kasvaessa sen heikkoudet alkavat tulla esiin. TypeScript pyrkii puuttumaan näihin ongelmiin tarjoamalla ohjelmoijalle moduulijärjestelmän, luokat, rajapinnat ja staattisen tyypityksen \parencite[]{understanding-typescript}.

Ohjelmointikielen modulaarisuudella tarkoitetaan sitä, että muuttujat, funktiot, luokat ja muut vastaavat ohjelmointikielen perusrakenteet ovat olemassa vain moduulien sisällä ellei niitä erikseen esitellä muille moduuleille \parencite[]{typescript-modules}.
- TODO: lisää pohdintaa modulaarisuuden edut?

Perinteisesti JavaScript on käyttänyt uudestikäytettävien komponenttien rakentamiseen funktioita ja prototyyppi-pohjaista perintää. Iso osa nykyohjelmoijista ei kuitenkaan ole tottunut käyttämään edellä mainittuja keinoja vaan nykyisin suosituin lähestymistapa uudelleenkäytettävyyteen on luokkien käyttäminen. \parencite[]{typescript-classes}
- TODO: luokkien edut vrt. js?


- rajapintojen edut vrt. js?
- staattisen tyypityksen edut vrt. js?

\section{Käytettävyyden perusteet}

Kaikki ihmisen tuottamat esineet ja asiat täytyy suunnitella. Vaikka ihmiset ovat suorittaneet aktiivista suunnittelua esihistoriallisista ajoista saakka, on tietoisten suunnitteluprosessien tutkimus verrattain tuoretta. Nykyinen suunnittelun kenttä voidaan jakaa karkeasti kolmeen eri osaan: teollinen suunnittelu, käytettävyyssuunnittelu ja kokemussuunnittelu \parencite[]{norman-doet}. Kommunikaatiossa avustavan sovelluksen suunnittelussa tulisi olla erityisen kiinnostuneita käytettävyyssuunnitelusta.

Hyvin suunniteltu sovellus on miellyttävä käyttää ja ohjaa käyttäjää käyttämään sovellusta oikealla tavalla. Tämä on erityisen tärkeää sovelluksissa, joita on tarkoitus käyttää useasti päivittäin.

- tyypillisimmät käytettävyystekijät DoET:sta tai muusta perusteoksesta käyttäen muitakin lähteitä

\subsection{Käytettävyys ja autismi}

Autismi vaikuttaa monella tapaa ihmisen kykyyn havainnoida ympäristöä ja sen myötä kykyyn käyttää sovelluksia. Iso-Britannian The National Autistic Society listaa verkkosivuillaan autistisille ihmisille sopivien verkkosivujen visuaalisesta toteutuksen päävaatimukset \parencite[]{autism-friendly-websites}. Listauksessa painotetaan erityisesti visuaalisen ilmeen selkeyttä, staattisuutta ja yksiselitteisyyttä. Lisäksi koekäyttäjien roolia korostetaan.

Symbolien käyttöä tulisi välttää TODO:lähde

Yksi varsin vähän tutkittu seikka on autismin vaikutus ihmisen suosikkiväreihin. Grandgeorgen ja Masatakan \parencite[]{color-preference-autism} mukaan autistiset lapset pitävät erityisesti vihreästä väristä. Keltaista ja ruskeaa tulisi välttää. Edellä mainitun tutkimuksen otoskoko oli kuitenkin pieni, joten tuloksia ei voi yleistää. Lapset pitävät erityisesti pääväreistä.

- etsi lisää lähteitä, ainakin .txt-tiedostossa oli jotain vanhoja

\chapter{Kommunikaatiossa avustavan sovelluksen suunnittelu ja toteutus}

\section{Ohjelmistotekninen näkökulma}
Tekninen kuvaus ohjelmasta. Verrataan tätä kirjallisuuden suosituksiin.

Sovellus koostuu valikkorakenteesta, korttien luontinäkymistä ja kommunikaationäkymästä.

Valikot

Ionicin standardielementeistä luotu.

Korttien luontinäkymä

Voidaan tehdä erikokoisia kortteja. Käytetään Papu.net-sivuston vapaata kuvapankkia.

Kommunikaationäkymä

Valitaan kortti. Kirjataan viesti symboleja painamalla.

\section{Käytettävyysnäkökulma}

Kerrotaan käytettävyyden näkökulmasta. Erityisesti erikoisryhmien käytettävyys nostetaan esille.

Tutkimuksessa toteutettu sovellus on tarkoitettu erityisesti autistisille käyttäjille.

\chapter{Toteutuneen sovelluksen arviointi}

Kuvien käsittely ja tallentaminen haastavaa. Progressiiviset sovellukset kohtalaisen hyviä tähän käyttötarkoitukseen eli yksinkertaisiin mobiilikeskeisiin sovelluksiin.


\chapter{Pohdinta}

Mitä aineistoanalyysi kertoo tutkimushypoteesista? Tukevatko tulokset teorialuvussa esitettyjä näkemyksiä? Vastasiko tutkimusmetodi odotuksia? Voiko tapaustutkimuksen pohjalta tehdä yleistyksiä?

Pohdintaluvussa harjoitetaan itsenäistä ajattelua. Sen voi periaatteessa kirjoittaa ilman lähdekirjallisuutta, sillä tarkastelun kohteena on aiemmin kirjoitettamasi materiaali. Nyt tutkimuksen tuloksia on aika arvioida kriittisesti. Niitä verrataan aiempaan tutkimukseen ja pohditaan, mitä uutta tutkimus paljastaa aiheesta ja miten tulokset suhteutuvat aiempaan tutkimukseen – tukevatko vai ovatko ristiriidassa? Jos analyysivaiheessa on ollut vaikeuksia, niitä ei pidä lakaista maton alle; niiden puntaroiminen kertoo tutkimuksen läpinäkyvyydestä ja tulosten luotettavuudesta. 

\chapter{Yhteenveto}

Tutkielman viimeinen luku on Yhteenveto.  Sen on hyvä olla lyhyt; siinä todetaan, mitä tutkielmassa esitetyn nojalla voidaan sanoa johdannon väitteen totuudesta tai tutkimuskysymyksen vastauksesta. Yhteenvedossa tuodaan myös esille tutkielman heikkoudet (erityisesti tekijät, jotka heikentävät tutkielman tulosten luotettavuutta), ellei niitä ole jo aiemmin tuotu esiin esimerkiksi Pohdinta-luvussa. Tässä luvussa voidaan myös tuoda esille, mitä tutkimusta olisi tämän
tutkielman tulosten valossa syytä tehdä seuraavaksi.

Jos Yhteenveto alkaa pitkittyä, se kannattaa jakaa kahtia niin, että tulosten tulkinta otetaan omaksi Pohdinta-luvukseen, jolloin Yhteenvedosta tulee varsin lyhyt ja lakoninen.

Yhteenvedon jälkeen tulee \string\printbibliography-komennolla laadittu lähdeluettelo ja sen jälkeen mahdolliset liitteet.

%\printbibliography{gradubibtex}

\printbibliography

%\bibliography{gradubibtex}{}
%\bibliographystyle{plain}

\appendix

\end{document}
