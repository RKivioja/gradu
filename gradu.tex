\documentclass[utf8]{gradu3}

\usepackage{graphicx}
\usepackage{amsmath}
\usepackage{booktabs}
\usepackage[backend=biber]{biblatex}
\usepackage[dvipsnames]{xcolor}

% HUOM! Tämän tulee olla viimeinen \usepackage koko dokumentissa!
\usepackage[bookmarksopen,bookmarksnumbered,linktocpage]{hyperref}

\addbibresource{gradubibtex.bib}

\begin{document}

\title{Puheen ongelmista kärsiville tarkoitetun kommunikointisovelluksen toteuttaminen Ionic-kehyksellä}
\translatedtitle{Puheen ongelmista kärsiville tarkoitetun kommunikointisovelluksen toteuttaminen Ionic-kehyksellä}
\studyline{Ohjelmistotekniikka}
\avainsanat{%
  Ionic,
  AAC,
  WWW-sovellukset,
  käytettävyys
}
\keywords{%
  Ionic,
  AAC,
  web applications,
  usability
}
\tiivistelma{%
  Tutkielman tiivistelmä on tyypillisesti lyhyt esitys, jossa kerrotaan tutkielman taustoista, tavoitteesta, tutkimusmenetelmistä, saavutetuista tuloksista, tulosten tulkinnasta ja johtopäätöksistä. Tiivistelmän tulee olla niin lyhyt, että se, englanninkielinen abstrakti ja muut metatiedot mahtuvat kaikki samalle sivulle.
}
\abstract{%
  Tiivistelmä englanniksi.
}

\author{Roope Kivioja}
\contactinformation{\texttt{roope.kivioja@gmail.com}}

\supervisor{Jonne Itkonen}
\supervisor{Jukka-Pekka Santanen}

\maketitle

\begin{thetermlist}
\item[Ionic] Ohjelmistokehys.
\end{thetermlist}

\mainmatter

\chapter{Johdanto}

\colorbox{YellowGreen}{// Ohjepituus 5 sivua. (Varmaan oikeasti 1-3 sivua.)}

\colorbox{YellowGreen}{// TODO: Tähän tulee johdanto.}

\chapter{Kommunikointisovelluksen rakentamiseen tarvittavia taustatietoja ja menetelmiä}
\colorbox{YellowGreen}{// Ohjepituus 10-15 sivua.}

\colorbox{YellowGreen}{// TODO: Lyhyt siltaava kappale johdannosta teoriakappaleeseen}

\colorbox{YellowGreen}{// TODO: kirjoita omaksi kappaleeksi:}

- Taustat, tarpeet ja tavoitteet:

- progressiivisten WWW-sovellusten tutkiminen

- progressiivisten WWW-sovellusten sopiminen avusteisen kommunikaation mobiilisovelluksien tuottamiseen

-> tavoite: Ionic-kehyksen toiminnan tutkiminen

-> tavoite: Ionic-kehyksen avulla tehdyn avusteisen kommunikaation sovelluksen vertaaminen ohjelmistoteknisiin ja käytettävyyspohjaisiin suosituksiin kehittäjän näkökulmasta

-> tavoite: Ionic-kehyksen avulla tehdyn avusteisen kommunikaation sovelluksen soveltuminen autistien kätettävyystarpeisiin suosituksien pohjalta

\section{Avusteinen kommunikaatio}

On olemassa useita eri sairauksia ja kehityshäiriöitä, joiden johdosta henkilön kyky muodostaa puhetta voi hetkellisesti tai pysyvästi heikentyä. Puheen muodostamisen ongelmista kärsivä henkilö saattaa joutua turvautumaan arkipäivän kommunikaatiossa erilaisiin apuvälineisiin kommunikoidakseen muiden ihmisten kanssa. Yleisesti näitä viestintämenetelmiä kuvaamaan tarkoitettu termi on \textbf{puhetta tukeva ja korvaava kommunikaatio} (engl. \textit{Augmentative and Alternative Communication, lyh. AAC}).

Puhetta tukeva ja korvaava kommunikaatio voidaan jakaa kahtia: avustamattomaan ja avusteiseen. \textbf{Avustamattomalla puhetta tukevalla ja korvaavalla kommunikaatiolla} tarkoitetaan kommunikaatiota, jossa ei tarvita apuvälineitä. Viittomakieli on yksi esimerkki avustamattomasta puhetta tukevasta ja korvaavasta kommunikaatiosta, mutta myös ihmisen elekieltä voidaan pitää avustamattomana puhetta tukevana ja korvaavana kommunikaationa. 

\textbf{Avustettu puhetta tukeva ja korvaava kommunikaatio} tarkoittaa puolestaan kommunikaatiota, jossa käytetään jotain apuvälinettä. Apuvälineenä voi olla esimerkiksi valokuvia, kommunikaatiotaulu tai elektroninen laite. Tämän perusteella avustettu puhetta tukevaa ja korvaava kommunikaatio voidaan jakaa vielä eteenpäin kahdeksi ryhmäksi: matalan teknologian ja korkean teknologian puhetta tukevaan ja korvaavaan kommunikaatioon. Käyttäjä ei välttämättä käytä vain yhtä edellä mainituista tyypeistä. \parencite[]{AAC-conditional-use}

Puhetta tukevaa ja korvaavaa kommunikaatiota voidaan tarjota puheen muodostamisen ongelmista kärsiville myös tietoteknisten sovellusten avulla, joiden kautta käyttäjä kirjoittaa joko suoraan tekstiä tai kommunikoi valitsemalla symboleja. Esimerkiksi autistisilla käyttäjillä on tyypillisesti hyvin henkilö- ja yksityiskohtaisia tarpeita puhetta tukevalle ja korvaavalle kommunikaatiolle, joten tietoteknisten sovellusten suhteellisen helppo muokattavuus puoltaa niiden käyttöä perinteisempien puhetta ja kommunikointia korvaavien apuvälineiden sijaan.

\label{AAC-symbols}
Puhetta tukevat ja korvaavat kommunikointisovellukset käyttävät yleisimmin symboleja. Usein symbolit ryhmitellään kortteihin, joita on helppo vaihtaa tilanteeseen sopivaksi \label{AAC-cards}. Autisteille tyypillistä on vahva visuaalis-avaruudellinen hahmotuskyky, joten tutkimuksen mukaan piirroksiin ja valokuviin liitetyt merkitykset ovat tälle ryhmälle luontevin tapa kommunikoida. Vaughnin ja Hornerin tutkimuksessa  \parencite[]{concrete-versus-verbal} Karl-nimisen autistisen koehenkilön haastava käytös ja aggressio vähenivät, kun pelkästään verbaalisesti annettujen ruokavaihtoehtojen rinnalle tuotiin kuvat ruoka-annoksista. Symbolipohjaiselle puhetta tukevalle ja korvaavalle kommunikaatiolle on siis sekä tutkimuspohjaista näyttöä että käytännön kokemuksiin perustuvaa kannustetta.

Puhetta tukevaan ja korvaavaan kommunikointisovellukseen voidaan liittää myös symboleita lukeva ääniominaisuus. Ääniominaisuus mahdollistaa kommunikoinnin näköyhteyden ulkopuolelle, vähentää symbolien tulkitsijan läsnäolon pakollisuutta ja helpottaa pidempien viestien rakentamista puhetta tukevassa ja korvaavassa kommunikointisovelluksessa. \parencite[]{AAC-interventions}

\colorbox{YellowGreen}{// TODO: ankkuroi edellinen kappale paremmin tekstiin}

Esimerkiksi ääniominaisuuden lisäämisellä avustetusta puhetta tukevasta ja korvaavasta kommunikaatiosta voidaan tehdä \textbf{monimodaalista}. Monimodaalisuus tarkoittaa usean eri kommunikaatiotavan kautta tapahtuvaa kommunikaatiota. Monimodaalinen kommunikaatio voi tapahtua eri tapojen kautta yhtäaikaisesti tai peräkkäin. Puhetta tukevasta ja korvaavasta kommunikaatiosta kannattaa tehdä monimodaalista useista eri syistä. Ensinnäkin, suurin osa kaikesta kommunikaatiosta on monimodaalista, sillä toiselle ihmiselle puhuttaessa on tavallista selkeyttää sanomaa ilmein ja elein. Toiseksi, puhetta tukevaa ja korvaavaa kommunikaatiota käyttävä henkilön tulee useasti kommunikoida muiden puhetta tukevaa ja korvaavaa kommunikaatiota käyttävien henkilöiden kanssa, jolloin vaihtoehdoista on hyötyä. Kolmas merkittävä syy on se, että eri puhetta tukevissa ja korvaavissa kommuunikaatiotavoissa on vahvuuksia ja heikkouksia, joten eri kommunikaatiotapoja sekoittamalla voidaan korvata yksittäisen kommunikaatiotavan heikkouksia. \parencite[]{AAC-conditional-use}

\colorbox{YellowGreen}{// TODO: mitä ongelmia} 

\section{Progressiiviset WWW-sovellukset}

\textbf{Progressiiviset sovellukset} (engl. \textit{Progressive Web Applications, lyh. PWAs}) ovat selaimessa ajettavia WWW-sovelluksia, joiden ulkoasu määrittyy alustakohtaisesti niin, että niiden ulkoasu on mahdollisimman yhdenmukainen laitteen natiivien sovellusten kanssa. Selainpohjaisuuden takia progressiiviset sovellukset pystyvät käyttämään tarjolla olevia sovellusympäristön resursseja joustavasti sen sijaan, että ne itse määrittäisivät vaatimukset. Termi on verrattain tuore ja vakiintumaton, sillä esimerkiksi Ionic-kehyksen dokumentaatiossa käytetään myös termiä \textbf{hybridisovellus} (engl. \textit{hybrid application}).

Progressiivisten sovellusten ilmeisin hyöty on alustariippumattomuus. Samaa sovellusta voidaan käyttää kaikissa ympäristöissä, jotka tukevat WWW-selaimia. Eri versioita samasta sovelluksesta ei tarvitse kehittää ja ylläpitää erikseen, joten progressiivisten sovellusten avulla voidaan säästää kalliita työresursseja. Hyvä esimerkki progressiivisten sovellusten käyttöä tukevasta markkinatilanteesta on nykyisten älytelevisioiden kirjava tarjonta, sillä valmistajien omien käyttöjärjestelmien lisäksi muunmuassa Apple, Amazon ja Roku kehittävät televisioon liitettäviä laitteita, joiden avulla käyttäjä voi ajaa erilaisia sovelluksia. Näiden kaikkien laitteiden tukeminen olisi hyvin vaikeaa perinteisten sovellusten avulla. \parencite[]{frankston-pwa}

Tällä hetkellä erityisesti Google panostaa progressiivisten sovellusten kehittämiseen Chrome-selaimen kehitysympäristön yhteydessä. Googlen \parencite[]{google-pwa-marketing} mukaan progressiiviset sovellukset tarjoavat perinteisiin WWW-sovelluksiin verrattuna enemmän luotettavuutta, käytettävyyttä ja monipuolisempaa sisältöä. Yrityksen mukaan luotettavuus paranee, sillä progressiiviset sovellukset pystyvät tarjoamaan sisältöä myös ilman verkkoyhteyttä. Google myös väittää, että progressiivisten sovellusten kohdalla käytettävyyttä parantaa nopeammin käyttäjän komentoihin vastaava käyttöliittymä ja sovellusmaisuus puolestaan parantaa käyttäjän immersiota.

Googlen Chrome-selain on rakennettu avoimen lähdekoodin Chromium-selaimen päälle. Yksi merkittävimmistä Chromium-selaimeen pohjautuvista progressiivisten WWW-sovellusten kehitykseen tarkoitetuista kehyksistä on Electron. Electron on Microsoftin nykyisin omistaman GitHubin kehittämä ja ylläpitämä. Esimerkkejä Electron-sovelluksista ovat Discord, Slack ja Visual Studio Code. On mahdollista, että Microsoftin päätös muuttaa Edge-selain Chromium-pohjaiseksi johtui ainakin osittain progressiivisten WWW-sovellusten vaatimista ominaisuuksista kuten tehokkaammasta muistinhallinnasta.

Twitter, AliExpress ja Lancôme ottivat vuonna 2017 käyttöön progressiiviset sovellukset ja julkaistut tulokset ovat olleet positiivisia. Twitter onnistui uuden progressiivisen sovelluksensa ansiosta lisäämään sivupäivityksiä 65\% käyttäjäsessiota kohden, lisäämään lähetettyjen Twitter-viestien määrää 75\% ja vähentämään käytön lopettamista 20\%. Lisäksi uusi sovellus käytti vähemmän kuin 3\% natiivin sovelluksen vaatimasta muistitilasta ja vähensi datan käyttöä 70\%. Datan käytön määrän vähentyminen on erityisen merkittävää, koska Twitter arvioi, että vuonna 2017 45\% sen sisällöstä ladattiin 2G-verkon läpi. \parencite[]{beginners-guide-pwa}

AliExpress ja Lancôme ovat molemmat verkkokauppoja, joilla on ollut ongelmia mobiiliverkkokauppojensa tehokkuuden kanssa. Progressiiviseen sovellukseen siirtymällä AliExpress lisäsi uusien asiakkaiden myyntitapahtumien määrää 104\%:lla ja Lancôme 17\%:lla. Lisäksi AliExpress tuplasi sivunlatausten määrän sekä kasvatti sessioiden pituutta 74\%:lla. Lancôme puolestaan onnistui lisäämään iOS-sessioiden määrää 53\%:lla. Edellä mainitut tulokset eivät kuitenkaan ole suoraan yleistettävissä yleisemmin progressiivisten sovellusten vaikutuksiin, sillä niiden otoskoko on erittäin pieni. \parencite[]{beginners-guide-pwa}

Progressiivisten sovellusten käyttö ei kuitenkaan ole ongelmatonta. Ensinnäkin, koska progressiivisten WWW-sovellusten hyödyntämät teknologiset kehitysaskeleet ja niiden käyttöön kannustavat taloudelliset tekijät ovat verrattain tuoreita, vaaditaan kehittäjiltä paljon uusien asioiden opettelua sekä vakiintumattomien työkalujen ja kehysten käyttämistä kehitystyössä. Toinen merkittävä haaste voi olla tiedon tallentaminen, sillä selaimen toimintaan pohjautuvalle sovellukselle ei ole varattu tietoturvan takia samoja oikeuksia kuin natiiveille sovelluksille. Kolmas ongelma voi tulla eteen käytettävyydessä, sillä selaimessa toimivan sovelluksen vaatimat resurssit saattavat hidastaa sovelluksen toimintaa. \parencite[]{pwa-design-challenges}

Myös progressiivisten sovellusten keskusmuistinkäyttöä on kritisoitu. Progressiiviset sovellukset vaativat perinteisiä sovelluksia enemmän muistia, koska niitä ajetaan selaimen päällä. On vaikea arvioida, kuinka suuri osa muistinkäytöstä johtuu sovelluksen toteutuksesta ja kuinka suuri osa itse ohjelmistokehyksestä, mutta hyvin yksinkertaisissakin testeissä on saatu tuloksia, jotka viittaavat moninkertaiseen keskusmuistinkäyttöön natiiveihin sovelluksiin verrattuna \parencite[]{electron-memory-usage}.

\colorbox{YellowGreen}{// - TODO: tähän omaa kriittistä pohdintaa}

Progressiiviset WWW-sovellukset sopivat siis parhaiten tilanteisiin, joissa vähän muistia vaativan sovelluksen halutaan tukevan useita eri alustoja, mutta joissa tavallinen WWW-sivu ei riitä. Avusteisen kommunikaation sovelluksien vaatimukset osuvat hyvin yhteen progressiivisten WWW-sovellusten vaatimusten kanssa, sillä kuvakorttien näyttäminen ei ole muisti-intensiivistä, mutta toisaalta taas avusteisen kommunikaation sovellusten halutaan olevan helposti muovattavissa henkilökohtaisiin tarpeisiin sopiviksi sekä toimivan useilla eri alustoilla.

\section{Ionic-ohjelmistokehyksen rakenne ja ominaisuudet}

Ionic on yksi suosituimpia progressiivisten sovellusten kehittämiseen tarkoitetuista ohjelmistokehyksistä. Electronin ollessa työasemille suunnattu kehys, Ionic keskittyy mobiililaitteiden vaatimuksiin. Ionic on avointa lähdekoodia ja hyödyntää Apache Cordova -ympäristöä sekä suosituimpia frontend-ohjelmistokehyksiä. Aiemmin Ionic tuki vain Angular-ohjelmistokehystä, mutta Ionicin versiosta 4 lähtien kehittäjällä on mahdollisuus valita itse käyttämänsä frontend-kehys. Ionic on MIT-lisenssin alainen ja täten avointa lähdekoodia. Angular-ohjelmistokehys perustuu TypeScript-ohjelmointikieleen. Ionicin perusrakenne on kuvattuna kaaviossa \ref{fig:ionic-structure}. Kehyksenä Ionicin pääpaino on tarjota kehittäjälle oikealta näyttävä ja visuaalisesti toimiva sovellus. Sen ei ole tarkoitus korvata tyypillisiä JavaScript-kirjastoja vaan se toimii niiden tukena. \parencite[]{ionic-documentation}

Ionicia ylläpitää ja kehittää Ben Sperryn ja Max Lynch perustama samanniminen yritys. Avoimen lähdekoodin projektina Ionic-kehyksen kehitykseen ja ylläpitoon pääsee osallistumaan GitHub-sivuston välityksellä kuka tahansa. Ionicin viimeisin pääversio on 4.0.0. Viime aikoina Ionicin kehitystyössä painoarvoa on annettu erityisesti Ionic-kehyksen ja Angularin välisten riippuvuuksien vähentämiseen, jotta Ionicia voitaisiin käyttää muiden kehyksien kuten Reactin ja Vue.js:n yhteydessä.

\begin{figure}[h]\centering
  \includegraphics[height=9cm,keepaspectratio]{ionic-structure}
  \caption[Ionic-sovelluksen perusrakenne]
  {Ionic-sovelluksen perusrakenne.}
  \label{fig:ionic-structure}
\end{figure}

Ionic koostuu komponenteista. Ionicin komponentit ovat uudelleenkäytettäviä käyttöliittymäelementtejä, jotka toimivat sovelluksen käyttöliittymän rakennusosina. Niiden avulla sovellus näyttää yhtenäiseltä ja käyttöliittymän kehitystyö nopeutuu. Komponentit koostuvat HTML:stä, CSS:stä ja JavaScriptistä. Esimerkkejä tällaisista komponenteista ovat muunmuassa painikkeet, välilehdet ja erilaiset listat. Ionicin toteutuksessa on pyritty tukemaan näiden komponenttien mahdollisimman joustavaa räätälöintiä \parencite[]{ionic-documentation}.

Ionicin ulkoasu perustuu teemoihin. Myös teemat lisäävät sovelluksen ulkoasun yhtenäisyyttä. Teemat myös mukautuvat eri alustojen ulkoasustandardien mukaisiksi kehittäjän niin halutessa. Esimerkiksi Android- ja iOS-mobiilikäyttöjärjestelmille suunnattujen sovellusten ulkoasu poikkeaa toisistaan, jos käytetään Ionicin oletusteemoja. Teemojen käyttäminen parantaa sovelluksen käytettävyyttä tekemällä sen ulkoasusta ennustettavamman ja tutumman loppukäyttäjälle. Ionicin oletusteemat noudattavat Applen iOS-designperiaatteita sekä Googlen Material Design -määrityksiä.

\subsection{Apache Cordova}

Apache Cordova (aikaisemmin PhoneGap) on alunperin Nitobin kehittämä sovelluskehitysympäristö mobiililaitteille. Adobe osti Nitobin vuonna 2011 ja myöhemmin uudelleenjulkaisi Apache Cordovan avoimena lähdekoodina. Apache Cordova toimii WWW-sovelluskehyksien ja natiivien sovelluskehyksien välisenä linkityskerroksena \parencite[]{ionic-framework-hybrid}. Apache Cordovan avulla Ionic ja muut Apache Cordovan päälle rakennetut ohjelmistokehykset voivat toimia syvemmällä tasolla kuin tyypillinen WWW-sovellus, sillä niille tarjoutuu rajapinta suoraan natiiveihin sovelluskehyksiin. Näin progressiivisille WWW-sovelluksille tarjoutuu mahdollisuus käyttää muunmuassa mobiililaitteen kameraa ja GPS-paikannusta.

\subsection{Angular}

Angular on Googlen Angular Teamin ylläpitämä TypeScript-pohjainen käyttöliittymäkehys. Se on jatkoa aiemmin ilmestyneelle AngularJS-kehykselle. Alkuperäinen AngularJS-kehys julkistettiin vuonna 2010, ja se oli ensimmäinen suosittu ohjelmistokehys dynaamisten HTML-sivujen kehittämiseen mahdollistaen tehokkaamman yhden sivun WWW-sovellusten rakentamisen. Angular julkaistiin vuonna 2016, ja se on kokonaan uudelleenohjelmoitu versio AngularJS:stä.

AngularJS perustuu MVC-arkkitehtuuriin, kun taas uudempi Angular on komponenttipohjainen. Komponenttipohjainen arkkitehtuuri pyrkii tarjoamaan paremman uudelleenkäytettävyyden, luettavuuden, testattavuuden ja ylläpidettävyyden. Ohjelma jaetaan itsenäisiin komponentteihin, joita voidaan käyttää useasti ja niiden itsenäisyys helpottaa yksikkötestaamista. Itsenäiset komponentit ovat huomattavasti helpommin ymmärrettävissä ja niiden korvaaminen on joustavampaa, mikä parantaa ylläpidettävyyttä. \parencite[]{good-and-bad-angular} 

Angularin komponenttipohjainen rakenne perustuu kolmeen hiljattain ilmestyneeseen teknologiaan: Web-komponentteihin (engl. \textit{Web Components}), JavaScriptin ES2015-standardiin ja TypeScript-ohjelmointikieleen.

\textbf{Web-komponentit} on laaja-alainen termi, jolla tarkoitetaan neljää WWW-selaimissa yleistyvää standardia: \textbf{kustomoitavat elementit} (engl. \textit{custom elements}), \textbf{varjo-DOM} (engl. \textit{shadow DOM}), \textbf{mallit} (engl. \textit{templates}) ja \textbf{HTML-tuonti} (engl. \textit{HTML imports}). 

Kustomoitavat HTML-elementit ovat HTML-standardielementtien ulkopuolisia elementtejä, joita voidaan käyttää standardielementtien seassa. Kustomoitava HTML-elementti irroittaa komponentin sivun muista osista, joten se mahdollistaa komponentin eristämisen. Varjo-DOM on piiloitettu osa sivua, jolla on oma eristetty ympäristö skripteille, CSS-tyylitiedostoille ja HTML-elementeille. Varjo-DOM:in elementit ja tyylit eivät vaikuta varjo-DOM:in ulkopuolisiin alueisiin ja vastavuoroisesti muut sivun elementit ja tyylit eivät vaikuta varjo-DOM:in alaisiin osiin. Komponentti voi täten käyttää tätä eristettyä aluetta renderöintiin. 

Mallit ovat HTML-palasia, jotka eivät lataudu välittömästi HTML-sivun auetessa vaan ne voidaan aktivoida JavaScriptillä myöhemmin. Malleista on useita toteutuksia eri kehyksissä, mutta Web-komponentit standardisoivat mallien rakentamisen ja tarjoavat suoran tuen niiden hyödyntämiselle selaimessa. Malleja käyttämällä varjo-DOM:iin piilotetusta sisällöstä voidaan tehdä dynaamista. Viimeinen Web-komponenttien osa on HTML-tuonti. HTML-tuonnin avulla HTML-, CSS- ja JavaScript-tiedostoja voidaan ladata yhtenäisinä osina. Angular ei käytä HTML-tuontia, vaan se käyttää JavaScriptin moduulilatausta. \parencite[]{angular-6-by-example}

\subsection{TypeScript}

TypeScript on avoimen lähdekoodin ohjelmointikieli, jota ylläpitää Microsoft. TypeScript kääntyy suoritusvaiheessa JavaScriptiksi, ja se on tarkoitettu tukemaan JavaScript-ohjelmien kehitystä parantelemalla JavaScriptin ominaisuuksia. TypeScript sisältää ES 2015 -standardin mukaiset ominaisuudet sekä lisäksi se tarjoaa ohjelmoijan käyttöön tyypit ja koristelijat. Angular-ohjelmointikehyksen ohjelmointiin käytetään TypeScript-ohjelmointikieltä.

TypeScriptin on tarkoitus tarjota Microsoftin .NET-ympäristöön tottuneille ohjelmoijille oliolähtöisempi lähestymistapa JavaScript-kehitykseen \parencite[]{maharry-typescript}. JavaScript on suosittu ohjelmointikieli, mutta varsinkin ohjelmiston lähdekoodin määrän kasvaessa sen heikkoudet alkavat tulla esiin. TypeScript pyrkii puuttumaan näihin ongelmiin tarjoamalla ohjelmoijalle moduulijärjestelmän, luokat, rajapinnat ja staattisen tyypityksen \parencite[]{understanding-typescript}.

Ohjelmointikielen \textbf{modulaarisuudella} tarkoitetaan sitä, että muuttujat, funktiot, luokat ja muut vastaavat ohjelmointikielen perusrakenteet ovat olemassa vain moduulien sisällä, ellei niitä erikseen esitellä muille moduuleille \parencite[]{typescript-modules}. Tämä vähentää ohjelman eri osien välistä riippuvuutta toisistaan, mikä helpottaa muutosten tekemistä ohjelmistoon.

Perinteisesti JavaScript on käyttänyt uudestikäytettävien komponenttien rakentamiseen funktioita ja prototyyppipohjaista perintää. Iso osa nykyohjelmoijista ei kuitenkaan ole tottunut käyttämään edellä mainittuja keinoja, vaan nykyisin suosituin lähestymistapa uudelleenkäytettävyyteen on luokkien käyttäminen. Myös JavaScriptin oliotukea on pyritty parantelemaan sen viimeisimmissä versioissa. \parencite[]{typescript-classes}

Uudelleenkäytettävyyden lisäksi TypeScriptin luokkien avulla aikaan modulaarisia komponentteja, joita on helpompi ylläpitää ja skaalata. Virheiden löytämistä helpottaa se, että luokkien avulla ohjelmakoodin rajoista saadaan selkeämpiä. Skaalautuvuutta varten täytyy monesti pystyä korvaamaan vanhoja komponentteja uusilla ja luokkia käyttämällä myös tämä on helpompaa.

\textbf{Rajapintoja} käytetään kuvaamaan luokkien ominaisuuksia. TypeScriptissä rajapintoja käytetään tyyppien nimeämiseen ja niiden avulla voidaan tehdä olioiden välisiä sopimuksia ohjelmoijan itse laatiman ohjelmistokoodin sisällä sekä myös ohjelmoijan oman ohjelmistokoodin ja muiden kehittämien kirjastojen välillä. \parencite[]{typescript-interfaces} Esimerkiksi Swagger-niminen työkalu generoi .NET-palvelinkoodista valmiita rajapintoja, joita TypeScript-pohjaisen WWW-sovelluksen on helpompi käyttää. Rajapinnat ovat luonnollinen osa luokkiin perustuvaa ohjelmointia. Niiden avulla luokan toteutus voidaan eroittaa sen ulkopuolelle näkyvistä osista ja jakaa luokkia ryhmiin. Kuten luokkienkin kohdalla, rajapintojen käyttäminen voi parantaa ohjelman ylläpidettävyyttä, skaalautuvuutta ja osien uudelleenkäytettävyyttä.

JavaScriptistä poiketen TypeScriptin on staattisesti tyypitetty kieli. Tyypitys tarkoittaa sitä, että muuttujien ja olioiden tyypit täytyy määritellä ohjelmakoodissa. Dynaamisesti tyypitetyissä kielissä kuten JavaScriptissä muuttujien ja olioiden tyypit päätellään käännöksen aikana. Staattisen tyypityksen avulla ohjelmoijan on helpompi nähdä suoraan, että ohjelman tieto on oikean tyyppistä sitä käsitellessä. Iso osa tyyppivirheistä jää tällöin kiinni jo ennen kääntämistä kun taas JavaScriptilla ohjelmoidun ohjelman kohdalla näin ei käy. Toisaalta vahva tyypitys aiheuttaa lisätyötä tyyppimuunnosten takia.

\colorbox{YellowGreen}{// TODO: tyypityksestä lisää?}

\section{Käytettävyyden perusteet}

Kaikki ihmisen tuottamat esineet ja asiat täytyy suunnitella. Vaikka olemme suorittaneet aktiivista suunnittelua esihistoriallisista ajoista saakka, on tietoisten suunnitteluprosessien tutkimus verrattain tuoretta. Nykyinen suunnittelun kenttä voidaan jakaa karkeasti kolmeen eri osaan: teollinen suunnittelu, käytettävyyssuunnittelu ja kokemussuunnittelu \parencite[]{norman-doet}. Kommunikaatiossa avustavan sovelluksen suunnittelussa tulee olla erityisen kiinnostuneita käytettävyyssuunnitelusta.

Hyvin suunniteltu sovellus on miellyttävä käyttää ja ohjaa käyttäjää käyttämään sovellusta oikealla tavalla. Tämä on erityisen tärkeää sovelluksissa, joita on tarkoitus käyttää arkisissa tilanteissa, jotka vaativat reaaliaikaista reagointia muiden ihmisten kommunikointiin. Avustettua kommunikaatiota käyttävillä ryhmillä on myös omia käytettävyystarpeita, joiden huomioimista varten täytyy ymmärtää käytettävyyttä tavallista laajemmassa kontekstissa.

\label{general-usability-requirements}
Ohjelman käyttöliittymän \textbf{käytettävyys} (\textit{engl. usability}) on laadullinen määre, joka voidaan jakaa viiteen laadulliseen osa-alueeseen: \textbf{opittavuus} (\textit{engl. learnability}), \textbf{tehokkuus} (\textit{engl. efficiency}), \textbf{muistettavuus} (\textit{engl. memorability}), \textbf{virhealttius} (\textit{engl. errors}) ja \textbf{tyydyttävyys} (\textit{engl. satisfaction}). 

Opittavuus tarkoittaa sitä miten helppo käyttöliittymää on oppia käyttämään. Käyttöliittymän tehokkuus määrittyy siitä miten nopeasti käyttäjät voivat suorittaa ohjelman toimintoja sen jälkeen kun ohjelman käyttöliittymää on opittu käyttämään. Muistettavuus on määrite, jonka avulla arvioidaan miten nopeasti käyttäjä muistaa toiminnot ohjelman käyttämisen lopettamisen ja uudelleen aloittamisen jälkeen. Virhealttiudella tarkoitetaan käyttäjien tekemien virheiden määrää ja niiden laatua sekä kuinka helppoa niistä selviäminen on. Tyydyttävyys kertoo,  miten tyytyväinen käyttäjä on käyttöliittymään. On olemassa muitakin laadullisia määreitä kuten \textbf{käyttökelpoisuus} (\textit{engl. utility}), jolla määritellään ohjelman tarjoamien eri ominaisuuksien määrää ja vertaamista haluttuihin ominaisuuksiin. Käytettävyyttä ja käyttökelpoisuutta yhtä aikaa tarkistelemalla saadaan aikaan kuva ohjelman \textbf{hyödyllisyydestä} (\textit{engl. usefulness}). \parencite[]{usability-101}

\colorbox{YellowGreen}{// TODO: opittavuus}
Lyhyesti ilmaistuna, opittavuus tarkoittaa ohjelmiston kykyä opettaa käyttäjälleen oikea tapa käyttää ohjelmistoa. Opittavuutta vastaava termi on \textbf{löydettävyys} (\textit{engl. discoverability}). Opittavuutta lisääviä ominaisuuksia ovat muunmuassa muistettavuus, loogisuus, toistettavuus ja yhdenmukaisuus. Opittavuudeltaan hyvä ohjelma ilmaisee käyttäjälleen tehokkaasti mitä toiminnallisuuksia se sisältää, mitä eri toiminnot tarkoittavat ja kuinka eri toiminnallisuuksia käytetään. \parencite[]{improving-learnability}

\colorbox{YellowGreen}{// TODO: tehokkuus}
Tehokkuus on määritelmä tai mitta siitä, miten helposti ja nopeasti haluttu toiminto voidaan suorittaa tuttua käyttöliittymää käyttämällä. Tehokkuutta voidaan suoraan mitata esimerkiksi kulunutta aikaa tai välivaiheiden määrää mittaamalla. 

\colorbox{YellowGreen}{// TODO: muistettavuus}
Muistettavuus mittaa sitä miten helppo käyttäjän on käyttää ohjelmistoa sen jälkeen kun ohjelmiston edellisestä käyttökerrasta on kulunut aikaa. Muistettavuutta on hankala mitata suoraan tyypillisten käyttäjätutkimusmetodien avulla, mutta esimerkiksi Affordable Usability -sivuston \parencite[]{affordable-usability} mukaan sitä voidaan WWW-sovellusten yhteydessä tutkia erilaisten verkkosivuanalytiikkatyökalujen avulla.

\colorbox{YellowGreen}{// TODO: virhealttius}
Käyttäjien tekemien virheiden määrän minimointi on yksi käyttettävyyssuunnittelun tärkeimmistä tavoitteista. Don Normanin mukaan \parencite[]{norman-doet} käyttäjää ei juuri koskaan pitäisi syyttää tekemistään virheistä vaan suurin osa käyttövirheistä johtuu huonosta suunnittelusta.

\colorbox{YellowGreen}{// TODO: tyydyttävyys}
Tyydyttävyyttä ymmärtääkseen täytyy tietää, että käyttöliittymä ja käyttäjäkokemus ovat kaksi eri asiaa. Käyttäjäkokemus ja sen tyydyttävyys tai tyydyttämyys on seurausta useasta eri tekijästä. Visuaalinen ulkoasu on osatekijä käyttöliittymän tyydyttävyyttä arvioidessa, mutta jos käyttäjä ei löydä haluamiaan toimintoja, osa halutuista ominaisuuksista puuttuu tai ne on hankalta löytyää, ohjelmiston tyydyttävyys on matala. Tyydyttävyys on viidestä edellä listatusta käytettävyystekijästä kaikista subjektiivisin.

\subsection{Käytettävyys ja avusteinen kommunikaatio}

Avusteiseen kommunikaatioon liittyy omia käytettävyyshaasteita. Tarve avusteiseen kommunikaatioon voi johtua useista eri syistä, joten käyttäjäkirjo ja käyttäjien henkilökohtaisten tarpeiden määrä on suuri. Jo pelkästään tämän havainnon perusteella voidaan olettaa ohjelmiston muokattavuuden olevan tärkeää.

CP-vammaisten avusteista kommunikaatiota tutkineessa tutkimuksessa \parencite[]{classmate-aac-study} yksi merkittävä käytettävyystekijä on avusteista kommunikaatiota käyttävän henkilön hidas toiminta. Avusteista kommunikaatiota käyttävä henkilö ei välttämättä pysty reagoimaan nopeasti eri keskusteluaiheisiin tai muodostamaan riittävän paljon kommunikaatiota, jolloin keskustelukumppani joutuu arvaamaan, mitä avusteista kommunikaatiota käyttävä henkilö yrittää sanoa. Käytettävyyden näkökulmasta on myös huomioitava, että käyttäjällä on riittävästi aikaa suorittaa valintoja käyttöliittymässä ilman, että näkymä vaihtuu tai käyttöliittymäelementit muuttuvat.

\label{AAC-context-settings}
Comunicador-nimistä espanjankielisille avusteisen kommunikaation käyttäjille tarkoitettua ohjelmistoa käsittelevässä tutkimuksessa \parencite[]{graphic-communicator} nostetaan esiin kontekstiriippuvaisten valintojen tärkeys. Koska kommunikointi on usein vaivalloista avusteisten kommunikaation sovellusten kautta, avusteisen kommunikaation sovelluksessa on hyvä olla mahdollisuus ryhmitellä esivalmistellut sanat tai kuvat tilanteiden mukaan. Esimerkiksi työ-, harrastus- ja arkitilanteissa tarvitaan usein hyvin erilaisia sanavarastoja. Kontekstin valitsemisen lisäksi sanoja tai kuvia täytyy myös pystyä ryhmittelemään ja järjestelemään kontekstiryhmien sisällä.

\label{AAC-staticity}
Autismi vaikuttaa monella tapaa ihmisen kykyyn havainnoida ympäristöä ja sen myötä kykyyn käyttää sovelluksia. Iso-Britannian The National Autistic Society listaa verkkosivuillaan \parencite[]{autism-friendly-websites} autistisille ihmisille sopivien verkkosivujen visuaalisesta toteutuksen päävaatimukset. Listauksessa painotetaan erityisesti visuaalisen ilmeen selkeyttä, staattisuutta ja yksiselitteisyyttä. Lisäksi koekäyttäjien roolia korostetaan.

Ilmaiseksi tarjolla olevia avusteisen kommunikaation mobiilisovelluksia tutkineen tutkimuksen \parencite[]{autism-mobile-usability} mukaan autistisilla käyttäjillä on ominaisuus- ja käytettävyystarpeita, joita tarjolla olleissa sovelluksissa ei ole. Tutkimuksen mukaan mahdollisuus kuvien ottamiseen sekä äänen tallentamiseen on erittäin hyödyllinen ominaisuus avusteisia kommunikaatiosovelluksia käyttäville autisteille. Kuvat eri sijainneista ja ihmisistä auttavat päivittäisessä kommunikaatiossa huomattavasti. \label{AAC-photos}

\label{AAC-settings}
Kommunikaatiosovelluksessa tulisi myös olla mahdollisuus säätää ohjelman asetuksia hallintapaneelin kautta. Hallintapaneelin tulisi olla salasanasuojattu, jotta käyttäjä ei pääse vahingossa poistamaan tärkeitä asetuksia tai kuvia sovelluksesta. Yhteen näyttöön ei saa mahduttaa liian paljon informaatiota, sillä autismiin liittyy usein aistiyliherkkyyttä, mikä heikentää autistin kykyä ymmärtää isoja määriä informaatiota kerrallaan. Tämän takia esimerkiksi kommunikaatiosovelluksen korteilla olevien kuvien määrää tulisi pystyä säätämään sovelluksen hallintapaneelista. \label{AAC-cardsize}

\label{AAC-soundsynth}
Kommunikaatiosovellusten ääniominaisuus voi auttaa käyttäjää oppimaan sovelluksen käyttöä nopeammin. Äänisyntetisaattoreita Martin-nimisen pojan avulla tutkineessa tutkimuksessa \parencite[]{voca-efficacy} havaittiin, että Martin oppi muodostamaan uusia lauseita helpommin, kun avusteisen kommunikaation sovellus luki kirjoitetun tekstin ääneen. Tutkimuksen luotettavuutta kuitenkin heikentää se, että tutkimuksessa tutkittiin vain yhden koehenkilön oppimistuloksia.

\label{symbol-libraries}
Symbolien käyttöä sovelluksien käyttöliittymissä tekstin korvikkeena pidetään yleisesti hyvänä tapana säästää tilaa, vähentää lukemisen aiheuttamaa kognitiivista kuormaa sekä kiertää tarvetta tekstien kääntämiseen useille eri kielille. Autistisilla henkilöillä esiintyy kuitenkin useasti hahmotusongelmia, jotka haittaavat symbolien merkityksen ymmärtämistä. Autististen henkilöiden on keskimäärin haastavaa ymmärtää abstraktien ja vähän ikonisuutta sisältävien symbolien merkitystä. \label{AAC-abstract-symbols} Esimerkiksi palloa esittävä symboli on autisteille helppo ymmärtää, mutta abstraktien asioiden kuten "mene"-verbin yhdistäminen nuoli-symboliin voi olla haastavaa. \parencite[]{symbol-acquisition-autism}

\label{AAC-colors}
Yksi jonkin verran tutkittu seikka on autismin vaikutus ihmisen suosikkiväreihin. Grandgeorgen ja Masatakan \parencite[]{color-preference-autism} mukaan autistiset lapset vaikuttavat pitävän vihreästä väristä. Keltaista ja ruskeaa tulisi välttää. Edellä mainitun tutkimuksen otoskoko oli kuitenkin pieni, joten tuloksia ei voi yleistää. Grandgeorgen ja Masatakan mukaan lapset pitävät erityisesti pääväreistä.

\chapter{Kommunikointisovelluksen suunnittelu ja toteutus}
\colorbox{YellowGreen}{// Ohjepituus 10-15 sivua.}

\colorbox{YellowGreen}{// TODO: Siltaava kappale, jossa alustetaan aliotsikot}

\colorbox{YellowGreen}{// TODO: Mikä on FreeAAC?}

Tutkielmaa varten toteutettiin FreeAAC-niminen avusteisen kommunikaation ohjelmisto.

\section{Ohjelmistotekninen näkökulma}

FreeAAC-ohjelmiston ominaisuusvaatimukset johdettiin edellisissä kappaleissa mainittujen ominaisuuksien perusteella. Edellisissä kappaleissa lainattujen tutkimusten perusteella kommunikointisovelluksen tulisi olla korttipohjainen \hyperref[AAC-cards]{''1''}. Korttien tulisi koostua kuvista ja symboleista \hyperref[AAC-symbols]{''2''}. Kuvia tulisi olla mahdollista valita valmiista kuvapankista \hyperref[symbol-libraries]{''3''} tai ottaa itse laitteen kameralla \hyperref[AAC-photos]{''4''}. Korttien kokoa tulisi olla mahdollista säätää käyttäjäkohtaisesti \hyperref[AAC-cardsize]{''5''}.

Itse kommunikaationäkymässä pitäisi olla mahdollisuus valita eri kortteja nopeasti ja helposti kontekstiin sopivaksi. Ohjelmiston pitäisi pystyä lukemaan symboleilla ja kuvilla kirjoitettu viesti ääneen \hyperref[AAC-soundsynth]{''6''}. 

Kortteja ja muita asetuksia pitäisi päästä hallitsemaan hallintapaneelista. Hallintapaneelin tulisi olla salasanalla lukittava, jotta erityisryhmään kuuluva käyttäjä ei vahingossa säätäisi korttien tai muiden ohjelmiston osien asetuksia vääränlaisiksi \hyperref[AAC-settings]{''7''}. 

\colorbox{YellowGreen}{// TODO: Kenties liitteeksi loppuun?}
- Ominaisuusvaatimukset:

  * kommunikaatiokortit
  
  * kuvien valitseminen kortilta viestiin
  
  * viestin äänisynteesi
  
  * korttien luominen ja poistaminen
  
  * korttien koon määrittäminen
  
  * kortin vaihtaminen
  
  * kuvien valitseminen korteille valmiista kirjastosta
  
  * kuvien valitseminen korteille kuvia ottamalla
  
  * lukittava hallintapaneeli ja asetusten tallentaminen

\section{Kehitysympäristö ja -työkalut}

\colorbox{YellowGreen}{// TODO: Ionic CLI, Ionic Lab, npm, VS Code, ESLint}

\colorbox{YellowGreen}{// TODO: Kirjoita tämä kappale yhtenäiseksi}

\textbf{Ionic CLI} (Ionic Command Line Interface, \textit{suom. Ionic-komentorivikehoite}) on Ionicin tarjoama konsolisovellus, jolla voidaan nopeasti generoida, asentaa ja räätälöidä Ionic-sovellukseen liittyviä tiedostoja.

Ionic-sovellusta voidaan ajaa paikallisena testiversiona neljällä eri tapaa. Selaimessa, iOS- tai Android-simulaattorilla, mobiililaitteen selaimessa tai itsenäisenä sovelluksena puhelimessa. FreeAAC-ohjelmistoa testattiin pääosin selainversiona. Alustariippumattomuutta on osana mahdollistamassa Node.js -ympäristö.

\textbf{Node.js} on alustariippumaton runtime-ympäristö palvelinpuolen JavaScript-koodin suorittamiseen. Lähes kaikki JavaScript-pohjaiset kirjastot nojaavat Node.js:ään ja myös Ionic vaatii Node.js:n asennuksen. Node.js Package Manager eli \textbf{npm} on paketinhallintajärjestelmä JavaScript-kirjastoille. FreeAAC-ohjelmiston tarvitsemat ulkopuoliset paketit asennettiin npm:ää käyttäen.

FreeAAC-ohjelmiston kehittämiseen käytettiin \textbf{Visual Studio Code} -nimistä ohjelmistoympäristöä. Visual Studio Code on Microsoftin kehittämä ja ylläpitämä avoimen lähdekoodin Electron-pohjainen ohjelmistoympäristö. Angularin käyttämä TypeScript-ohjelmointikieli on myös Microsoftin kehittämä, joten Visual Studio Coden TypeScript-tukeen on panostettu. Visual Studio Code:n tukena käytettiin myös liitännäisiä kuten ESLint-analyysityökalua.

\textbf{ESLint} on koodianalyysityökalu, joka käy läpi ohjelmistoympäristössä olevaa ohjelmakoodia ja auttaa ohjelmoijaa ohjelmoimaan koodianalyysityökaluun valittujen asetusten mukaista ohjelmakoodia.

\section{Rakenne}

FreeAAC noudattaa Ionic-sovelluksien oletusrakennetta. Ionic-sovelluksissa kansiorakenteen ylimmälle tasolle sijoitetaan projektin asetuksia sisältävät tiedostot. Projektin asetustiedostoista ohjelmistokehitysprosessin kannalta tärkein on \texttt{package.json}. Tiedostossa listataan kaikki ohjelmiston käyttämät ulkopuoliset kirjastot ja Node.js:n paketinhallintajärjestelmä npm lataa sen perusteella oikeat versiot jokaisesta kirjastosta Ionic-sovelluksen käytettäväksi.

Seuraavalla kansiotasolla Ionicin oletusrakenneteessa projekti jakautuu kahteen pääkansioon: \texttt{Resources} ja \texttt{Src}. Ohjelmiston pikakuvakkeiden ja käynnistysruudun tarvitsemat mediatiedostot ovat \texttt{Resources}-kansiossa.

\texttt{Src}-kansiossa sijatsevat ohjelmiston lähdekoodia sisältävät tiedostot. \texttt{Src}-kansion päätasolla sijaitsevat \texttt{index.html}, \texttt{manifest.json} ja \texttt{service-worker.js}-tiedostot. \texttt{Index.html} on ensimmäinen HTML-tiedosto, jonka käynnistyvä Ionic-sovellus lukee. Se sisältää Ionic-sovelluksen juurikomponentin ja linkkejä sovelluksen yleisesti vaatimiin resursseihin kuten tyylitiedostoihin.

\texttt{Src}-kansiota alemmalla tasolla on oletusprojektissa viisi eri kansiota: \texttt{Assets}, \texttt{Classes}, \texttt{Pages}, \texttt{Providers} ja \texttt{Theme}.

\texttt{Assets}-kansioon ohjelmiston kehittäjä voi sijoittaa kuva- ja äänitiedostoja, joita ohjelman toiminnallisuudet vaativat. FreeAAC-sovelluksen kuvakirjasto sijaitsee kansiossa.

\texttt{Classes}-kansiossa sijaitsevat ohjelmiston käyttämät luokat. FreeAAC-ohjelmistossa kaksi käytössä olevaa luokkaa ovat \texttt{Card} ja \texttt{WordSymbol}. \texttt{Card}-luokka vastaa kommunikointisovelluksissa käytettäviä kortteja. Se sisältää tiedon kortin nimestä, sen koosta ja kortin sisältämistä kuvista ja symboleista. \texttt{WordSymbol}-luokka taas on representaatio korteilla olevien kuvien ja symbolien tietomallista. Se sisältää kuvan tai symbolin nimen sekä viittauksen kuvatiedostoon.

Ohjelmiston näkymät sijaitsevat \texttt{Pages}-kansiossa. FreeAAC-ohjelmistossa on seitsemän eri näkymää: Home, Main, CardCreate, CardDelete, Info, Options ja SelectSymbolModal. Angular-ohjelmistokehystä käyttävässä Ionic-sovelluksessa näkymät ovat HTML-tiedostoja, joihin on \textbf{sidottu} (\textit{engl. bind}) komponenttitiedostossa olevia muuttujia ja aliohjelmia. Lisäksi kansioon sijoitetaan yleensä moduulitiedosto, jossa sijaitsevat viittaukset tarvittaviin ulkopuolisiin kirjastoihin sekä muihin moduuleihin ja tyylitiedosto, jolla ohjelmiston ulkoasua voidaan säätää näkymäkohtaisesti.

\texttt{Theme}-kansiossa sijaitsevat ohjelmiston ulkoasuun vaikuttavat SCSS-tiedostot. Ionicin oletusprojektissa se sisältää \texttt{variables.scss}-tiedoston, joka asettaa projektille Ionicin oletusteeman. Kehittäjä voi itse vapaasti säätää ohjelmiston päätason ulkoasua tätä kautta.

\textbf{Riippuvuusinjektio} (engl. \textit{dependency injection}) on suunnittelumalli, jossa luokalle annetaan toinen luokka tai staattinen aliohjelma, joka tarjoaa luokalle uusia ominaisuuksia käyttöön. Riippuvuusinjektiossa luokka tai staattinen aliohjelma annetaan toiselle luokalle sen sijaan, että luokka itse hakisi luokan tai staattisen aliohjelman. Angular-ohjelmistokehyksessä riippuvuusinjektio toteutetaan tuomalla toinen luokka import-lauseella mukaan luokan lähdekoodiin ja julistamalla riippuvuusinjektio luokan konstruktorissa.

\texttt{Providers}-kansioon sijoitetaan ohjelmiston palveluluokat. \textbf{Palvelumalli} (engl. \textit{service pattern}) on suunnittelumalli, jossa useiden eri luokkien käyttämiä palveluita sijoitetaan samaan luokkaan yhdeksi palveluksi. Näin keskeisten palveluiden toteutus voidaan irroittaa itse luokkien toteutuksesta.

\colorbox{YellowGreen}{// TODO: DI:stä ja palvelumallista lisää?}

\section{Ohjelmiston näkymät}

Ionicin näkymiä käsitellään pinona. Ionic-sovelluksessa näkymästä toiseen siirrytään lisäämällä uusi näkymä pinon päällimmäiseksi käyttämällä Ionicin \texttt{NavController}-kontrolleriluokkaa. Näin näkymäsiirtymien historia säilyy ja käyttäjä voi palata edellisiin näkymiin helposti painamalla mobiililaitteen edellinen-painiketta. Tällöin viimeksi pinoon lisätty näkymä poistetaan pinosta.

\begin{figure}[h]\centering
  \includegraphics[height=7cm,keepaspectratio]{FreeAACViews}
  \caption[FreeAAC-sovelluksen näkymäpuu.]
  {FreeAAC-sovelluksen näkymäpuu.}
  \label{fig:FreeAACViews}
\end{figure}

FreeAAC:n Ionic-näkymissä sisältö on jaettu kahden eri päätason HTML-elementin sisälle: \texttt{ion-header} ja \texttt{ion-content}. \texttt{Ion-header}-lohkossa sijaitsevat näkymän ylälaitaan tuleva teksti, joka kertoo missä näkymässä ollaan tällä hetkellä. Lisäksi sinne voidaan sijoittaa valikkotoimintoja. \texttt{Ion-content}-lohkossa sijaitsee näkymän varsinainen pääsisältö.

Home-näkymä on FreeAAC-ohjelmiston ensimmäinen näkymä ja se sisältää ohjelmiston päävalikon. Päävalikossa sijaitsevat painikkeet, joita klikkaamalla käyttäjä pääsee keskustelunäkymään (Main), asetusnäkymään (Options) tai infonäkymään (Info). Home-näkymän komponenttitiedostossa ei ole muita kuin näkymän vaihtamiseen liittyviä aliohjelmia.

Ohjelmiston tärkein näkymä on Main-niminen näkymä, jossa sijaitsevat ohjelmiston varsinaiset kommunikointityökalut. Main-näkymän komponenttitiedosto sisältää viestilaatikon päivittämiseen sekä korttien tietojen lataamiseen tarvittavat aliohjelmat. Korttien lataamiseen käytetään \texttt{CardDataProvider}-luokkaa, joka on injektoitu mukaan komponenttiin sen konstruktorissa.

Options-näkymä on kokoomanäkymä ohjelmiston hallintaa varten. Options-näkymän kautta käyttäjä pääsee siirtymään korttien luontinäkymään (CardCreate) ja korttien poistonäkymään (CardDelete). 

Kortteja luodaan ohjelmiston CardCreate-näkymässä. CardCreate-näkymässä kortille voidaan valita nimi, koko ja liittää siihen kuvia kuvakirjastosta. Kaikki edellä mainitut ovat pakollisia ja käyttäjälle näytetään toast-virheilmoitus, jos jokin tieto puuttuu. Toast-virheilmoitukset käyttävät Ionicin \texttt{ToastController}-luokkaa. Kuvia valitaan erillisen modaalin kautta ja Ionic-sovelluksessa modaalien näyttämisestä vastaa \texttt{ModalController}-luokka.  Korttien tallentaminen tapahtuu \texttt{CardDataProvider}-luokan kautta.

Kuvien valintaa varten aukeava modaali sisältää SelectSymbolModal-näkymän. Näkymässä kuvan voi valita eri kategorioista. Kategoria valitaan pudotusvalikosta. SelectSymbolModal-näkymän komponenttitiedosto sisältää kuvan tallentamiseen ja modaalin sulkemiseen liittyvät aliohjelmat. FreeAAC-ohjelmiston kuvakirjasto koostuu Papunet-sivuston tarjoamasta Creative Commons -lisenssin alaisesta kuvapankista.

Kortteja voidaan poistaa erillisessä CardDelete-näkymässä. CardDelete-näkymä koostuu pudotusvalikosta, josta voidaan valita mikä kortti halutaan poistaa sekä vahvistuspainikkeesta. Korttilistaus ladataan \texttt{CardDataProvider}-luokan avulla ja myös poistokomento kulkee samaa reittiä pitkin.

Info-näkymässä listataan FreeAAC-sovellukseen liittyviä tietoja. Info-näkymä on staattinen eikä siinä ole toiminallisuuksia.

FreeAAC:lla on kaksi palveluluokkaa: \texttt{CardDataProvider} ja \texttt{ImageDataProvider}. \texttt{CardDataProvider}-luokan tehtävä on tarjota näkymäkomponenteille tietoa olemassa olevista korteista. \texttt{ImageDataProvider}-luokka huolehtii ohjelmiston kuvakirjaston eri kuvien toimittamisesta näkymäkomponenteille.

\colorbox{YellowGreen}{// TODO: tallennustila}

\section{Käytettävyysnäkökulma}

FreeAAC-ohjelmiston käytettävyydelle asetettiin tutkimustiedon perusteella joukko vaatimuksia. FreeAAC-ohjelmiston tulisi noudattaa yleisiä käytettävyysvaatimuksia \hyperref[general-usability-requirements]{''1''}, mutta myös sen lisäksi ottaa huomioon erityisryhmien, erityisesti autististen käyttäjien, käytettävyysvaatimukset huomioon. FreeAAC-ohjelmiston värityksen tulisi olla pääväreihin perustuva, selkeä ja sen näkymissä tulisi olla riittävästi kontrastia \hyperref[AAC-colors]{''2''}. 

Abstraktien symbolien käyttämistä tulisi välttää, sillä autistiset käyttäjät eivät välttämättä ymmärrä niiden merkitystä samalla tavalla kuin eritysryhmiin kuulumattomat käyttäjät\hyperref[AAC-abstract-symbols]{''3''}. FreeAAC-ohjelmiston käytettävyyttä tosielämän nopeaa reagointia vaativissa tilanteissa lisäisi mahdollisimman perinpohjainen mahdollisuus räätälöintiin \hyperref[AAC-context-settings]{''4''}. Näkymistä tulisi myös tehdä mahdollisimman staattisia ja yksinkertaisia \hyperref[AAC-staticity]{''5''}.

\colorbox{YellowGreen}{// TODO: Kenties liitteeksi loppuun?}
- Ominaisuusvaatimukset:

  * yleiset käytettävyysvaatimukset
  * selkeiden päävärien käyttö ja riittävästi kontrastia
  * abstraktien symbolien välttäminen
  * mahdollisuus kontekstiin sopivien asetusten säätämiseen
  * näkymien staattisuus ja yksinkertaisuus

\colorbox{YellowGreen}{// TODO: Yleiset käytettävyysratkaisut.}
FreeAAC-sovellus käyttää Ionicin oletuselementtejä ja oletusteemaa. Oletuselementit ja oletusteema noudattavat Applen iOS-designperiaatteita sekä Googlen Material Design -määrityksiä.  

FreeAAC-sovelluksen elementit ovat riittävän isoja, jotta niiden valitseminen ja toiminnallisuuksien ymmärtäminen on helpompaa kun ohjelmistoa ajetaan pieninäyttöisillä mobiililaitteilla. Suuret elementit myös rajoittavat yhteen näkymään mahtuvien ominaisuuksien määrää, mikä auttaa pitämään näkymät riittävän yksinkertaisina käyttäjälle.

\colorbox{YellowGreen}{// TODO: Kosketusnäyttöeleiden käytettävyys.}
Sovelluksen käyttämiseen ei tarvita eritysryhmille mahdollisesti motorisesti haastavia eleitä kuten pyyhkäisyjä. Erityisesti autisteilla on usein ongelmia hienomotoristen liikkeiden suorittamisessa, vaikka autististen käyttäjien yksilökohtaiset erot ovat tässäkin suuria \parencite[]{motor-skills-autism}.

FreeAAC-ohjelmiston valikot on toteutettu perinteisellä tavalla painikkeita hyödyntämällä. Monessa mobiilisovelluksessa navigointi tapahtuu niin sanotun \textit{"hampurilaisvalikon"} avulla, mutta autistisille käyttäjille tarkoitetussa ohjelmistossa abstraktien symbolien käyttö voi olla ongelmallista.

\colorbox{YellowGreen}{// TODO: Käytettävyys näkymittäin.}
Valikkonäkymien lisäksi käytettävyysratkaisuja on tehty erityisesti kommunikaationäkymässä (Main) ja korttien asetusnäkymissä (CardCreate ja CardDelete). Kommunikaationäkymä koostuu kolmesta eri komponentista: viestilaatikko, korttivalinta ja symbolivalinta. Viestilaatikkossa näkyvät käyttäjän valitsemat symbolit järjestyksessä. Käyttäjä voi poistaa symboleita yksi kerrallaan viimeisestä symbolista alkaen. Symbolikortti valitaan pudotusvalikosta, josta löytyvät käyttäjän kortinluonti-näkymässä tekemät kortit. Sivun alalaidassa näkyvät kortin symbolit, joista käyttäjä voi valita haluamansa symbolit viestilaatikkoon.

Korttien asetusnäkymät on erotettu toisistaan näkymien yksinkertaistamisen takia.


\chapter{Toteutuneen sovelluksen arviointi}
\colorbox{YellowGreen}{// Ohjepituus 15-20 sivua.}

\colorbox{YellowGreen}{// TODO: Pitäisikö yhdistää edelliseen osaan?}

FreeAAC-ohjelmiston prototyyppi sisältää suurimman osan tarpeellisista ominaisuuksista ja siitä pystyttiin toteuttamaan toimiva sekä koekäyttöön sopiva versio tämän tutkielman aikarajoitteiden puitteissa. Voidaan todeta, että ainakin perustasolla progressiiviset WWW-sovellukset vaikuttavat lupaavalta ja riittävän kypsältä teknologialta yksinkertaisten mobiilisovellusten toteuttamiseen.

FreeAAC-ohjelmiston ohjelmakoodin määrä on maltillinen verrattuna ominaisuusvaatimuksiin, mikä parantaa ohjelmiston ylläpidettävyyttä ja testattavuutta. Ohjelmistossa on myös hyvin vähän sisäisiä riippuvuuksia, koska Angular-ohjelmistokehyksen sekä TypeScript-ohjelmointikielen avulla ohjelmistosta saatiin helposti tehtyä modulaarinen. 

\colorbox{YellowGreen}{// TODO: Ohjelmakoodin määrä (rivit, metodit, luokat?), Halstead tms. ohjelmistometriikka?}
\colorbox{YellowGreen}{// TODO: Sisäiset riippuvuudet.}

\colorbox{YellowGreen}{// TODO: Mittaa muistinkäyttöä ja vertaa johonkin.}

Näkymien HTML-formaatti ei juurikaan aiheuttanut ohjelmistoteknisiä haasteita tai käytettävyysongelmia. Ainoastaan symbolien vapaa asemointi kortilla osoittautui sen verran haastavaksi tehtäväksi, että sitä ei toteutettu. HTML-kieli on alunperin tarkoitettu dokumenttien hierarkkiseksi ja rakenteeliseksi kuvauskieleksi, joten HTML-elementtien kelluvaa asemointia toisiinsa nähden ei ole helpoin toteuttaa sen avulla. Kelluvien elementtien toteuttaminen on kyllä mahdollista oikeanlaisten CSS-määritysten avulla, mutta tämä vaatisi erillisen asemointipalvelun rakentamista ohjelmistoon sekä CSS-luokkien lisäilyä ja poistamista reaaliajassa. Perinteisessä mobiilisovelluksessa elementtien vapaa asemointi saattaisi olla helpompaa.

\colorbox{YellowGreen}{// TODO: Tarkenna tallennustilaan liittyvät asiat.}
Kenties haastavin osuus FreeAAC-ohjelmiston toteuttamisessa oli tietojen tallentaminen. Kuten luvussa xx todettiin, progressiivisille WWW-sovelluksille ei ole luontevaa tallennustilaa, sillä selainten oikeuksien määrä on hyvin tarkasti rajoitettua tietoturvasyistä. Kuvat voidaan liittää mukaan sovellukseen, mutta korttien asetuksien tallentaminen ei ole yksinkertaista. Natiivisovelluksilla on suora pääsy rajattuun massamuistitilaan, mutta WWW-sovelluksissa tiedot tulee tallentaa muualle.

\chapter{Pohdinta}
\colorbox{YellowGreen}{// Ohjepituus 5-10 sivua.}

\colorbox{YellowGreen}{-- Mitä aineistoanalyysi kertoo tutkimushypoteesista? Tukevatko tulokset teorialuvussa esitettyjä näkemyksiä? Vastasiko tutkimusmetodi odotuksia? Voiko tapaustutkimuksen pohjalta tehdä yleistyksiä?

Pohdintaluvussa harjoitetaan itsenäistä ajattelua. Sen voi periaatteessa kirjoittaa ilman lähdekirjallisuutta, sillä tarkastelun kohteena on aiemmin kirjoitettamasi materiaali. Nyt tutkimuksen tuloksia on aika arvioida kriittisesti. Niitä verrataan aiempaan tutkimukseen ja pohditaan, mitä uutta tutkimus paljastaa aiheesta ja miten tulokset suhteutuvat aiempaan tutkimukseen – tukevatko vai ovatko ristiriidassa? Jos analyysivaiheessa on ollut vaikeuksia, niitä ei pidä lakaista maton alle; niiden puntaroiminen kertoo tutkimuksen läpinäkyvyydestä ja tulosten luotettavuudesta. --}

\chapter{Yhteenveto}
\colorbox{YellowGreen}{// Ohjepituus 3-5 sivua.}

\colorbox{YellowGreen}{-- Tutkielman viimeinen luku on Yhteenveto.  Sen on hyvä olla lyhyt; siinä todetaan, mitä tutkielmassa esitetyn nojalla voidaan sanoa johdannon väitteen totuudesta tai tutkimuskysymyksen vastauksesta. Yhteenvedossa tuodaan myös esille tutkielman heikkoudet (erityisesti tekijät, jotka heikentävät tutkielman tulosten luotettavuutta), ellei niitä ole jo aiemmin tuotu esiin esimerkiksi Pohdinta-luvussa. Tässä luvussa voidaan myös tuoda esille, mitä tutkimusta olisi tämän tutkielman tulosten valossa syytä tehdä seuraavaksi.

Jos Yhteenveto alkaa pitkittyä, se kannattaa jakaa kahtia niin, että tulosten tulkinta otetaan omaksi Pohdinta-luvukseen, jolloin Yhteenvedosta tulee varsin lyhyt ja lakoninen.

Yhteenvedon jälkeen tulee \string\printbibliography-komennolla laadittu lähdeluettelo ja sen jälkeen mahdolliset liitteet. --}

%\printbibliography{gradubibtex}

\printbibliography

%\bibliography{gradubibtex}{}
%\bibliographystyle{plain}

\appendix

\end{document}
