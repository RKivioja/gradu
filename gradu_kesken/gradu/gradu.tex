\documentclass[utf8]{gradu3}

\usepackage{graphicx}
\usepackage{amsmath}
\usepackage{booktabs}

% HUOM! Tämän tulee olla viimeinen \usepackage koko dokumentissa!
\usepackage[bookmarksopen,bookmarksnumbered,linktocpage]{hyperref}

\addbibresource{gradubibtex.bib} % Lähdetietokannan tiedostonimi

\begin{document}

\title{Puheen ongelmista kärsiville tarkoitetun kommunikointisovelluksen toteuttaminen Ionic-kehyksellä}
\translatedtitle{Puheen ongelmista kärsiville tarkoitetun kommunikointisovelluksen toteuttaminen Ionic-kehyksellä}
\studyline{Ohjelmistotekniikka}
\avainsanat{%
  Ionic,
  AAC,
  www-sovellukset,
  käytettävyys
}
\keywords{%
  Ionic,
  AAC,
  web applications,
  usability
}
\tiivistelma{%
  Tämä kirjoitelma on esimerkki siitä, kuinka
  {gradu3}-tutkielmapohjaa käytetään.  Se sisältää myös
  käyttöohjeet ja tutkielman rakennetta koskevia ohjeita.

  Tutkielman tiivistelmä on tyypillisesti lyhyt esitys, jossa
  kerrotaan tutkielman taustoista, tavoitteesta, tutkimusmenetelmistä,
  saavutetuista tuloksista, tulosten tulkinnasta ja johtopäätöksistä.
  Tiivistelmän tulee olla niin lyhyt, että se, englanninkielinen
  abstrakti ja muut metatiedot mahtuvat kaikki samalle sivulle.
}
\abstract{%
  Tiivistelmä englanniksi.
}

\author{Roope Kivioja}
\contactinformation{\texttt{roope.kivioja@gmail.com}}

\supervisor{Jukka-Pekka Santanen}
\supervisor{Se toinen}

\maketitle

\begin{thetermlist}
\item[Ionic] Ohjelmistokehys.
\item[AAC] Assistive and augmented computing eli puhetta tukeva ja korvaava kommunikaatio.
\end{thetermlist}

\mainmatter

\chapter{Johdanto}

Tähän tulee johdanto.

Yksi merkittävimmistä ongelmista nykyisissä AAC-sovelluksissa on niiden heikko saatavuus eri laitteille. (lähde, viittaus omaan haastatteluun?) Tämä oli suurin syy progressiivisen sovelluksen kehittämiseksi.

\chapter{Taustatietoa}

- TODO: Kerrotaan yleisesti taustoista (mihin avustavaa tietotekniikkaa tarvitaan, miten sitä käytetään, mitä ongelmia jne.)

On olemassa useita eri sairauksia ja kehityshäiriöitä, joiden johdosta henkilön puheen tuottamisen kyky voi hetkellisesti tai pysyvästi heikentyä. Tällöin näistä ongelmista kärsivä henkilö saattaa joutua turvautumaan arkipäivän kommunikaatiossa erinäisiin tietoteknisiin sovelluksiin, joiden kautta käyttäjä kirjoittaa joko suoraan tekstiä tai kommunikoi valitsemalla symboleja. Yleisesti näitä viestintämenetelmiä kuvaamaan tarkoitettu termi on "puhetta tukeva ja korvaava kommunikaatio" (engl. Augmentative and Alternative Communication, lyh. AAC).

- TODO: Kerrotaan yleisesti PWA-sovelluksista/hybridisovelluksista (mihin pohjautuu, miksi käytetään, miksi juuri tähän projektiin)

Progressiiviset sovellukset ovat selaimessa ajettavia WWW-sovelluksia, joiden ulkoasu on optimoitu laitteelle niin, että ne eivät eroa natiiveista sovelluksista. Progressiivisuudella tarkoitetaan sovellusta, joka pystyy käyttämään tarjolla olevan sovellusympäristön resursseja joustavasti sen sijaan, että se itse määrittäisi vaatimukset. Termi on verrattain tuore ja vakiintumaton, sillä esimerkiksi Ionic-kehyksen dokumentaatiossa käytetään myös termiä "hybridisovellus".

Idea natiivien sovellusten rakentamisesta webteknologioiden päälle ei ole uusi. Muutamia esimerkkejä idean käytöstä ovat Adobe AIR - ja Electron -sovellukset. (koko edellinen \parencite[]{escaping-tabs})

Tällä hetkellä erityisesti Google panostaa progressiivisten sovellusten kehittämiseen Chrome-selaimen kehitysympäristön yhteydessä. Googlen mukaan progressiiviset sovellukset tarjoavat perinteisiin WWW-sovelluksiin verrattuna enemmän luotettavuutta, käytettävyyttä ja monipuolisempaa sisältöä. Luotettavuus paranee, sillä progressiiviset sovellukset pystyvät tarjoamaan sisältöä myös ilman verkkoyhteyttä. Käytettävyyttä parantaa nopeammin käyttäjän komentoihin vastaava käyttöliittymä. Sovellusmaisuus puolestaan parantaa käyttäjän immersiota. TODO: tähän omaa pohdintaa (edellinen kappale \parencite[]{google-pwa-marketing}))

Ionic on yksi suosituimpia progressiivisten sovellusten tuottamiseen suunnittelluista kehyksistä. Se on avointa lähdekoodia ja hyödyntää Apache Cordova -ympäristöä sekä AngularJS -ohjelmistokehystä.

\section{Ionic-kehys}

Kerrotaan PWA-sovelluksista/hybridisovelluksista (mihin pohjautuu, miksi käytetään, miksi juuri tähän projektiin)

Nykyisten AAC-sovellusten ongelma on, että niitä voidaan käyttää vain yhdellä laitteella (lähde?), joten progressiivisten sovellusten käyttäminen voisi parantaa tilannetta.

Progressiivisten sovellusten käyttö ei kuitenkaan ole ongelmatonta. Ensinnäkin, koska kyse on uudesta suuntauksesta ohjelmistokehitysessä, vaaditaan kehittäjiltä paljon uusien asioiden opettelua sekä vakiintumattomien työkalujen sekä kehysten käyttämistä kehitystyössä. Toinen merkittävä kompastuskivi voi olla tallentaminen, sillä selaimen toimintaan pohjautuvalle sovellukselle ei ole varattu luontevaa tallennustilaa. Kolmas ongelma voi tulla eteen käytettävyydessä, sillä selaimessa toimivan sovelluksen vaatimat resurssit saattavat hidastaa sovelluksen toimintaa. TODO: tähän omaa pohdintaa (edellinen kappale \parencite[]{pwa-design-challenges}))

\section{AAC-sovellukset}

Kerrotaan taustoista (mihin avustavaa tietotekniikkaa tarvitaan, miten sitä käytetään, mitä ongelmia jne.)

Puhetta tukeva ja korvaava kommunikaatio voidaan jakaa kahtia: avustamattomaan ja avusteiseen. Avustamattomalla AAC:lla tarkoitetaan puhetta tukevaa ja korvaavaa kommunikaatiota, jossa ei tarvita apuvälineitä. Viittomankieli on yksi esimerkki avustamattomasta AAC:sta, mutta myös ihmisen elekieltä voidaan pitää avustamattomana puhetta tukevana ja korvaavana kommunikaationa. Avustettu AAC tarkoittaa puolestaan kommunikaatiota, jossa käytetään jotain apuvälinettä. Apuvälineenä voi olla esimerkiksi valokuvia, kommunikaatiotaulu tai elektroninen laite. Tämän perusteella avustettu AAC voidaan jakaa vielä eteenpäin kahdeksi ryhmäksi: matalan teknologian ja korkean teknologian AAC:hen. Käyttäjä ei välttämättä käytä vain yhtä edellä mainituista AAC-tyypeistä.  TODO: jaa kappaleisiin, lähdeviittaus: (koko edellinen \parencite[]{AAC-conditional-use})

\chapter{Sovelluksen toteutus ja rakenne}

Kerrotaan mistä palasista ohjelma koostuu ja miten ne keskustelevat keskenään. Verrataan tätä kirjallisuuden suosituksiin.
Täysin tekninen kuvaus.

Sovellus koostuu valikkorakenteesta, korttien luontinäkymästä ja kommunikaationäkymästä.

\section{Valikot}

Ionicin standardielementeistä luotu.

\section{Korttien luontinäkymä}

Voidaan tehdä erikokoisia kortteja. Käytetään Papu.net-sivuston vapaata kuvapankkia.

\section{Kommunikaationäkymä}

Valitaan kortti. Kirjataan viesti symboleja painamalla.

\chapter{Toteutuksessa opittua}

Kerrotaan lyhyesti opituista asioista.

Tuli aivan hyvä sovellus.

\section{Ohjelmistotekniikan näkökulma}

Kerrotaan ohjelmistotekniikan näkökulmasta kaikki opittu.

Kuvien kanssa värkkääminen aika haastavaa. Progressiiviset sovellukset kohtalaisen hyviä tähän käyttötarkoitukseen eli yksinkertaisiin mobiilikeskeisiin sovelluksiin.

\section{Käytettävyyden näkökulma}

Kerrotaan käytettävyyden näkökulmasta. Erityisesti erikoisryhmien käytettävyys nostetaan esille.

Tutkimuksessa toteutettu sovellus on tarkoitettu erityisesti autistisille käyttäjille. Autismiin liittyvän aistiyliherkkyyden takia sovelluksen toteutuksessa käytettävyys oli isossa roolissa.

Autismi vaikuttaa monella tapaa ihmisen kykyyn havainnoida ympäristöä ja sen myötä kykyyn käyttää sovelluksia. Iso-Britannian The National Autistic Society listaa verkkosivuillaan autistisille ihmisille sopivien verkkosivujen visuaalisesta toteutuksen päävaatimukset \parencite[]{autism-friendly-websites}. Listauksessa painotetaan erityisesti visuaalisen ilmeen selkeyttä, staattisuutta ja yksiselitteisyyttä. Lisäksi koekäyttäjien roolia korostetaan.

Symbolien käyttöä tulisi välttää TODO:lähde

Yksi varsin vähän tutkittu seikka on autismin vaikutus ihmisen suosikkiväreihin. Grandgeorgen ja Masatakan \parencite[]{color-preference-autism} mukaan autistiset lapset pitävät erityisesti vihreästä väristä. Keltaista ja ruskeaa tulisi välttää. Edellä mainitun tutkimuksen otoskoko oli kuitenkin pieni, joten tuloksia ei voi yleistää. Lapset pitävät erityisesti pääväreistä.

\chapter{Johtopäätökset}

Johtopäätöksiä (pitäisikö miettiä miten saadaan edellisen kappaleen kanssa jaettua nätisti?)

\chapter{Yhteenveto}

Tutkielman viimeinen luku on Yhteenveto.  Sen on hyvä olla lyhyt;
siinä todetaan, mitä tutkielmassa esitetyn nojalla voidaan sanoa
johdannon väitteen totuudesta tai tutkimuskysymyksen vastauksesta.
Yhteenvedossa tuodaan myös esille tutkielman heikkoudet (erityisesti
tekijät, jotka heikentävät tutkielman tulosten luotettavuutta), ellei
niitä ole jo aiemmin tuotu esiin esimerkiksi Pohdinta-luvussa.  Tässä
luvussa voidaan myös tuoda esille, mitä tutkimusta olisi tämän
tutkielman tulosten valossa syytä tehdä seuraavaksi.

Jos Yhteenveto alkaa pitkittyä, se kannattaa jakaa kahtia niin, että
tulosten tulkinta otetaan omaksi Pohdinta-luvukseen, jolloin
Yhteenvedosta tulee varsin lyhyt ja lakoninen.

Yhteenvedon jälkeen tulee \string\printbibliography-komennolla
laadittu lähdeluettelo ja sen jälkeen mahdolliset liitteet.

\printbibliography

\appendix
\section{Liite-esimerkki}

Keskeneräisen tutkielman siirtäminen gradu2:sta gradu3:een ei ole
kovin vaikeata.  Aluksi on totta kai vaihdettava
\string\documentclass-komennossa gradu2 gradu3:ksi.  Komennon
optioista suurin osa on poistettava, koska niitä ei enää tueta;
ainoastaan merkistön ilmoittava optio jää jäljelle.  Mahdollinen
kandi-optio vaihdetaan optioksi bachelor.

Taulukossa~\ref{tbl:cmdchange} on lueteltu tarvittavat
komentovaihdokset.  Viiva tarkoittaa, ettei vastaavaa komentoa ole
lainkaan.  Huomaa erityisesti uudet komennot.

\begin{table}[h]\centering
  \begin{tabular}{ll}
    \toprule
    gradu2                 & gradu3  \\
    \midrule
    ---                    & \string\maketitle \\
    ---                    & \string\supervisor \\
    \string\acmccs         & --- \\
    \string\aine           & \string\subject\\
    \string\copyrightowner & --- \\
    \string\fulltitle      & --- \\
    \string\laitos         & \string\department\\
    \string\license        & --- \\
    \string\linja          & \string\studyline\\
    \string\paikka         & --- \\
    \string\setauthor      & \string\author\\
    \string\termlist       & thetermlist-ympäristö\\
    \string\tyyppi         & \string\type\\
    \string\yhteystiedot   & \string\contactinformation\\
    \string\yliopisto      & \string\university\\
    \string\ysa            & --- \\
    \bottomrule
  \end{tabular}
  \caption{Komentomuutokset gradu2:sta gradu3:een}
  \label{tbl:cmdchange}
\end{table}

Isoin työ voi aiheutua lähdeluettelon laatimistekniikan muuttumiseen
sopeutumisesta.

\section{Liite-esimerkki 2}

Aiemmin esiteltyjen lisäksi gradu3 tarjoaa seuraavat lisäominaisuudet:
\begin{itemize}
\item \LaTeXe:n vakio-optiot draft ja final toimivat.
\item Vaikka tutkielman suomenkielisyyttä ei tarvitse erikseen
  mainita, finnish-optio toimii.
\item \string\university-komennolla voit ilmoittaa tutkielman
  kotiyliopistoksi jonkin muun kuin Jyväskylän yliopiston.
\item  \string\department-komennolla voit ilmoittaa tutkielman
  kotilaitokseksi jonkin muun kuin Informaatioteknologian tiedekunnan.
\item \string\subject-komennolla voit ilmoittaa tutkielman
  oppiaineeksi jonkin muun kuin tietotekniikan.  Huomaa, että oppiaine
  tulisi suomenkielisissä tutkielmissa kirjoittaa genetiivimuodossa ja
  isolla alkukirjaimella (''Tietotekniikan''), englanninkielisissä
  tuktkielmissa in-preposition kanssa (''in Information Technology'').
\item \string\type-komennolla voit ilmoittaa tutkielman tyypin, jos se
  on jokin muu kuin pro gradu (oletus) tai kandidaatintutkielma
  (optiolla bachelor).
\item \string\setdate-komennolla voit asettaa päivämäärän
  haluamaksesi.  Anna komennolle kolme parametria -- päivä,
  kuukausi ja vuosi -- numeerisessa muodossa.
\item Ympäristöllä chapterquote voit laittaa luvun alkuun
  mietelauseen.  Sillä on yksi pakollinen parametri (lainauksen
  attribuutio).
\item Komento \string\graduclsdate\ sisältää käytössä olevan gradu3:n
  julkaisupäivämäärän ja \string\graduclsversion\ sen versionumeron.
\end{itemize}

\end{document}
