\documentclass[utf8]{gradu3}

\usepackage{graphicx}
\usepackage{amsmath}
\usepackage{booktabs}
\usepackage[backend=biber]{biblatex}

% HUOM! Tämän tulee olla viimeinen \usepackage koko dokumentissa!
\usepackage[bookmarksopen,bookmarksnumbered,linktocpage]{hyperref}

\addbibresource{gradubibtex.bib}

\begin{document}

\title{Puheen ongelmista kärsiville tarkoitetun kommunikointisovelluksen toteuttaminen Ionic-kehyksellä}
\translatedtitle{Puheen ongelmista kärsiville tarkoitetun kommunikointisovelluksen toteuttaminen Ionic-kehyksellä}
\studyline{Ohjelmistotekniikka}
\avainsanat{%
  Ionic,
  AAC,
  www-sovellukset,
  käytettävyys
}
\keywords{%
  Ionic,
  AAC,
  web applications,
  usability
}
\tiivistelma{%
  Tämä kirjoitelma on esimerkki siitä, kuinka
  {gradu3}-tutkielmapohjaa käytetään.  Se sisältää myös
  käyttöohjeet ja tutkielman rakennetta koskevia ohjeita.

  Tutkielman tiivistelmä on tyypillisesti lyhyt esitys, jossa
  kerrotaan tutkielman taustoista, tavoitteesta, tutkimusmenetelmistä,
  saavutetuista tuloksista, tulosten tulkinnasta ja johtopäätöksistä.
  Tiivistelmän tulee olla niin lyhyt, että se, englanninkielinen
  abstrakti ja muut metatiedot mahtuvat kaikki samalle sivulle.
}
\abstract{%
  Tiivistelmä englanniksi.
}

\author{Roope Kivioja}
\contactinformation{\texttt{roope.kivioja@gmail.com}}

\supervisor{Jukka-Pekka Santanen}
\supervisor{a}

\maketitle

\begin{thetermlist}
\item[Ionic] Ohjelmistokehys.
\item[AAC] Assistive and augmented computing eli puhetta tukeva ja korvaava kommunikaatio.
\end{thetermlist}

\mainmatter

\chapter{Johdanto}

Tähän tulee johdanto.

Yksi merkittävimmistä ongelmista nykyisissä AAC-sovelluksissa on niiden heikko saatavuus eri laitteille. (lähde, viittaus omaan haastatteluun?) Tämä oli suurin syy progressiivisen WWW-sovelluksen kehittämiseksi.

\chapter{Kommunikaatiossa avustavan sovelluksen rakentamiseen tarvittavia taustatietoja ja menetelmiä}

\section{Avusteinen kommunikaatio}

- TODO: Kerrotaan yleisesti taustoista (mihin avustavaa tietotekniikkaa tarvitaan, miten sitä käytetään, mitä ongelmia jne.)

On olemassa useita eri sairauksia ja kehityshäiriöitä, joiden johdosta henkilön puheen tuottamisen kyky voi hetkellisesti tai pysyvästi heikentyä. Puheen tuottamisen ongelmista kärsivä henkilö saattaa joutua turvautumaan arkipäivän kommunikaatiossa erilaisiin apuvälineisiin kommunikoidakseen muiden ihmisten kanssa. Yleisesti näitä viestintämenetelmiä kuvaamaan tarkoitettu termi on "puhetta tukeva ja korvaava kommunikaatio" (engl. Augmentative and Alternative Communication, lyh. AAC). 

Puhetta tukevaa ja korvaavaa kommunikaatiota usein tarjotaan puheen tuottamisen ongelmista kärsiville myös tietoteknisten sovellusten avulla, joiden kautta käyttäjä kirjoittaa joko suoraan tekstiä tai kommunikoi valitsemalla symboleja. Esimerkiksi autistisilla käyttäjillä on tyypillisesti hyvin henkilö- ja yksityiskohtaisia tarpeita puhetta tukevalle ja korvaavalle kommunikaatiolle, joten tietotekniset sovellukset sopivat hyvin ratkaisemaan ongelmia, joita ei voida ratkaista perinteisemmin menetelmin.

Kerrotaan taustoista (mihin avustavaa tietotekniikkaa tarvitaan, miten sitä käytetään, mitä ongelmia jne.)

Puhetta tukeva ja korvaava kommunikaatio voidaan jakaa kahtia: avustamattomaan ja avusteiseen. Avustamattomalla AAC:lla tarkoitetaan puhetta tukevaa ja korvaavaa kommunikaatiota, jossa ei tarvita apuvälineitä. Viittomankieli on yksi esimerkki avustamattomasta AAC:sta, mutta myös ihmisen elekieltä voidaan pitää avustamattomana puhetta tukevana ja korvaavana kommunikaationa. Avustettu AAC tarkoittaa puolestaan kommunikaatiota, jossa käytetään jotain apuvälinettä. Apuvälineenä voi olla esimerkiksi valokuvia, kommunikaatiotaulu tai elektroninen laite. Tämän perusteella avustettu AAC voidaan jakaa vielä eteenpäin kahdeksi ryhmäksi: matalan teknologian ja korkean teknologian AAC:hen. Käyttäjä ei välttämättä käytä vain yhtä edellä mainituista AAC-tyypeistä.  TODO: jaa kappaleisiin, lähdeviittaus: (koko edellinen \parencite[]{AAC-conditional-use})

Tyypillisesti AAC-sovelluksessa kommunikaatio tapahtuu symbolien avulla. Autisteille tyypillistä on vahva visuaalis-avaruudellinen hahmotuskyky, joten tutkimuksen mukaan piirroksiin ja valokuviin liitetyt merkitykset ovat tälle ryhmälle tyypillinen tapa kommunikoida. Vaughnin ja Hornerin tutkimuksessa  \parencite[]{concrete-versus-verbal} Karl-nimisen autistisen koehenkilön haastava käytös ja aggressio vähenväti kun pelkästään verbaalisesti annettujen ruokavaihtoehtojen rinnalle tuotiin kuvat ruoka-annoksista. Symbolipohjaiselle AAC-kommunikoinnille on siis sekä tutkimuspohjaista näyttöä, että käytännön kokemuksiin perustuvaa kannustetta.

AAC-sovellukseen voidaan liittää myös symboleita lukeva ääniominaisuus. Ääniominaisuus mahdollistaa kommunikoinnin näköyhteyden ulkopuolelle, vähentää symbolien tulkitsijan läsnäolon pakollisuutta ja helpottaa pidempien viestien rakentamista AAC-sovelluksessa.\parencite[]{AAC-interventions}

\section{Progressiiviset WWW-sovellukset}

Kerrotaan PWA-sovelluksista/hybridisovelluksista (mihin pohjautuu, miksi käytetään, miksi juuri tähän projektiin)

- TODO: Kerrotaan yleisesti PWA-sovelluksista/hybridisovelluksista (mihin pohjautuu, miksi käytetään, miksi juuri tähän projektiin)

Progressiiviset sovellukset ovat selaimessa ajettavia WWW-sovelluksia, joiden ulkoasu on optimoitu laitteelle niin, että ne eivät eroa natiiveista eli sovelluksista. Progressiivisuudella tarkoitetaan sovellusta, joka pystyy käyttämään tarjolla olevan sovellusympäristön resursseja joustavasti sen sijaan, että se itse määrittäisi vaatimukset. Termi on verrattain tuore ja vakiintumaton, sillä esimerkiksi Ionic-kehyksen dokumentaatiossa käytetään myös termiä "hybridisovellus".

Tällä hetkellä erityisesti Google panostaa progressiivisten sovellusten kehittämiseen Chrome-selaimen kehitysympäristön yhteydessä. Googlen mukaan progressiiviset sovellukset tarjoavat perinteisiin WWW-sovelluksiin verrattuna enemmän luotettavuutta, käytettävyyttä ja monipuolisempaa sisältöä. Luotettavuus paranee, sillä progressiiviset sovellukset pystyvät tarjoamaan sisältöä myös ilman verkkoyhteyttä. Käytettävyyttä parantaa nopeammin käyttäjän komentoihin vastaava käyttöliittymä. Sovellusmaisuus puolestaan parantaa käyttäjän immersiota. TODO: tähän omaa pohdintaa (edellinen kappale \parencite[]{google-pwa-marketing}))

Idea natiivien sovellusten rakentamisesta selainalustoja hyödyntämällä ei ole uusi. Muutamia esimerkkejä idean käytöstä ovat Adobe AIR - ja Electron -sovellukset. (koko edellinen \parencite[]{escaping-tabs}) Twitter, AliExpress ja Lancôme ottivat vuonna 2017 käyttöön progressiiviset sovellukset ja julkaistut tulokset ovat rohkaisevia. Twitter onnistui uuden progressiivisen sovelluksensa ansiosta lisäämään sivupäivityksiä 65\% per sessio, lisäämään lähetettyjen Twitter-viestien määrää 75\% ja vähentämään käytön lopettamista 20\%. Lisäksi uusi sovellus käytti vähemmän kuin 3\% natiivin sovelluksen vaatimasta tilasta ja vähensi datan käyttöä 70\%. Datan käytön määrän vähentyminen on erityisen merkittävää, koska Twitter arvioi, että 45\% sen sisällöstä ladataan 2G-verkon läpi. Sekä AliExpress että Lancôme ovat molemmat verkkokauppoja, joilla on ollut ongelmia mobiiliverkkokauppojensa tehokkuuden kanssa. Progressiiviseen sovellukseen siirtymällä AliExpress lisäsi uusien asiakkaiden myyntitapahtumien määrää 104\%:lla ja Lancôme 17\%:lla. Lisäksi AliExpress tuplasi sivunlatausten määrän sekä kasvatti sessioiden pituutta 74\%:lla. Lancôme puolestaan onnistui lisäämään iOS-sessioiden määrää 53\%:lla. \parencite[]{beginners-guide-pwa})  TODO: omaa pohdintaa

Nykyisten AAC-sovellusten ongelma on, että niitä voidaan käyttää vain yhdellä laitteella (lähde?), joten progressiivisten sovellusten käyttäminen voisi parantaa tilannetta.

Progressiivisten sovellusten käyttö ei kuitenkaan ole ongelmatonta. Ensinnäkin, koska kyse on uudesta suuntauksesta ohjelmistokehitysessä, vaaditaan kehittäjiltä paljon uusien asioiden opettelua sekä vakiintumattomien työkalujen sekä kehysten käyttämistä kehitystyössä. Toinen merkittävä kompastuskivi voi olla tallentaminen, sillä selaimen toimintaan pohjautuvalle sovellukselle ei ole varattu luontevaa tallennustilaa. Kolmas ongelma voi tulla eteen käytettävyydessä, sillä selaimessa toimivan sovelluksen vaatimat resurssit saattavat hidastaa sovelluksen toimintaa. TODO: tähän omaa pohdintaa (edellinen kappale \parencite[]{pwa-design-challenges}))

\section{Ionic}

Ionic on yksi suosituimpia progressiivisten sovellusten tuottamiseen suunnitelluista kehyksistä. Ionic on avointa lähdekoodia ja hyödyntää Apache Cordova -ympäristöä sekä Angular-ohjelmistokehystä. Angular-ohjelmistokehitys perustuu TypeScript-ohjelmointikieleen.

\subsection{Angular}
\subsection{TypeScript}

\section{Käytettävyyden perusteet}

\chapter{Kommunikaatiossa avustavan sovelluksen suunnittelu ja toteutus}

\section{Ohjelmistotekninen näkökulma}
Kerrotaan mistä palasista ohjelma koostuu ja miten ne keskustelevat keskenään. Verrataan tätä kirjallisuuden suosituksiin.
Täysin tekninen kuvaus.

Sovellus koostuu valikkorakenteesta, korttien luontinäkymistä ja kommunikaationäkymästä.

Valikot

Ionicin standardielementeistä luotu.

Korttien luontinäkymä

Voidaan tehdä erikokoisia kortteja. Käytetään Papu.net-sivuston vapaata kuvapankkia.

Kommunikaationäkymä

Valitaan kortti. Kirjataan viesti symboleja painamalla.

\section{Käytettävyysnäkökulma}

Kerrotaan käytettävyyden näkökulmasta. Erityisesti erikoisryhmien käytettävyys nostetaan esille.

Tutkimuksessa toteutettu sovellus on tarkoitettu erityisesti autistisille käyttäjille. Autismiin liittyvän aistiyliherkkyyden takia sovelluksen toteutuksessa käytettävyys oli isossa roolissa.

Autismi vaikuttaa monella tapaa ihmisen kykyyn havainnoida ympäristöä ja sen myötä kykyyn käyttää sovelluksia. Iso-Britannian The National Autistic Society listaa verkkosivuillaan autistisille ihmisille sopivien verkkosivujen visuaalisesta toteutuksen päävaatimukset \parencite[]{autism-friendly-websites}. Listauksessa painotetaan erityisesti visuaalisen ilmeen selkeyttä, staattisuutta ja yksiselitteisyyttä. Lisäksi koekäyttäjien roolia korostetaan.

Symbolien käyttöä tulisi välttää TODO:lähde

Yksi varsin vähän tutkittu seikka on autismin vaikutus ihmisen suosikkiväreihin. Grandgeorgen ja Masatakan \parencite[]{color-preference-autism} mukaan autistiset lapset pitävät erityisesti vihreästä väristä. Keltaista ja ruskeaa tulisi välttää. Edellä mainitun tutkimuksen otoskoko oli kuitenkin pieni, joten tuloksia ei voi yleistää. Lapset pitävät erityisesti pääväreistä.

\chapter{Toteutuneen sovelluksen arviointi}

Kuvien kanssa värkkääminen aika haastavaa. Progressiiviset sovellukset kohtalaisen hyviä tähän käyttötarkoitukseen eli yksinkertaisiin mobiilikeskeisiin sovelluksiin.

Tuli aivan hyvä sovellus.

\chapter{Pohdinta}

Mitä aineistoanalyysi kertoo tutkimushypoteesista? Tukevatko tulokset teorialuvussa esitettyjä näkemyksiä? Vastasiko tutkimusmetodi odotuksia? Voiko tapaustutkimuksen pohjalta tehdä yleistyksiä?

Pohdintaluvussa harjoitetaan itsenäistä ajattelua. Sen voi periaatteessa kirjoittaa ilman lähdekirjallisuutta, sillä tarkastelun kohteena on aiemmin kirjoitettamasi materiaali. Nyt tutkimuksen tuloksia on aika arvioida kriittisesti. Niitä verrataan aiempaan tutkimukseen ja pohditaan, mitä uutta tutkimus paljastaa aiheesta ja miten tulokset suhteutuvat aiempaan tutkimukseen – tukevatko vai ovatko ristiriidassa? Jos analyysivaiheessa on ollut vaikeuksia, niitä ei pidä lakaista maton alle; niiden puntaroiminen kertoo tutkimuksen läpinäkyvyydestä ja tulosten luotettavuudesta. 

\chapter{Yhteenveto}

Tutkielman viimeinen luku on Yhteenveto.  Sen on hyvä olla lyhyt; siinä todetaan, mitä tutkielmassa esitetyn nojalla voidaan sanoa johdannon väitteen totuudesta tai tutkimuskysymyksen vastauksesta. Yhteenvedossa tuodaan myös esille tutkielman heikkoudet (erityisesti tekijät, jotka heikentävät tutkielman tulosten luotettavuutta), ellei niitä ole jo aiemmin tuotu esiin esimerkiksi Pohdinta-luvussa. Tässä luvussa voidaan myös tuoda esille, mitä tutkimusta olisi tämän
tutkielman tulosten valossa syytä tehdä seuraavaksi.

Jos Yhteenveto alkaa pitkittyä, se kannattaa jakaa kahtia niin, että tulosten tulkinta otetaan omaksi Pohdinta-luvukseen, jolloin Yhteenvedosta tulee varsin lyhyt ja lakoninen.

Yhteenvedon jälkeen tulee \string\printbibliography-komennolla laadittu lähdeluettelo ja sen jälkeen mahdolliset liitteet.

%\printbibliography{gradubibtex}

\printbibliography

%\bibliography{gradubibtex}{}
%\bibliographystyle{plain}

\appendix

\end{document}
